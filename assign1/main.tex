
                                         
\let\negmedspace\undefined
\let\negthickspace\undefined
\documentclass[journal,14pt,onecolumn]{IEEEtran}
\usepackage{cite}
\usepackage{amsmath,amssymb,amsfonts,amsthm}
\usepackage{algorithmic}
\usepackage{graphicx}
\usepackage{textcomp}
\usepackage{xcolor}
\usepackage{txfonts}
\usepackage{listings}
\usepackage{enumitem}
\usepackage{mathtools}
\usepackage{gensymb}
\usepackage{comment}
\usepackage[breaklinks=true]{hyperref}
\usepackage{tkz-euclide} 
\usepackage{listings}
\usepackage{gvv}                                        
%\def\inputGnumericTable{}                                 
\usepackage[latin1]{inputenc} 
\usepackage{color}                                            
\usepackage{array}                                            
\usepackage{longtable}                                       
\usepackage{calc}                                             
\usepackage{multirow}                                         
\usepackage{hhline}                                           
\usepackage{ifthen}                                           
\usepackage{lscape}
\usepackage{tabularx}
\usepackage{array}
\usepackage{float}

\newcommand{\begin{eqnarray}
\newcommand{\EEQA}{\end{eqnarray}
\newcommand{\define}{\stackrel{\triangle}{=}}

\begin{document}
\bibliographystyle{IEEEtran}


\title{CHAPTER 8- Circles}
\author{EE24BTECH11021 - Eshan Ray}
\maketitle

\renewcommand{\thefigure}{\theenumi}
\renewcommand{\thetable}{\theenumi}


\section*{Section-A \sbrak{JEE ADVANCED/IIT-JEE}}
\section*{A:FILL IN THE BLANKS}
\begin{enumerate}
    
  

	\item If A and B are points in the plane such that        $\frac{PA}{PB}=K$\brak{constant} for all P on a given circle, then the value of K cannot be equal to

		\hfill\brak{1982-2 Marks}
\item The points of intersection of the line $4x-3y-10=0$ and the circle $x^{2}+y^{2}-2x+4y-20=0$ are    

	\hfill\brak{1983-2Marks}
\item The lines $3x-4y+4=0$ and $6x-8y-7=0$ are tangents to the same circle. The radius of the circle is

	\hfill\brak{1984-2 Marks}
\item Let $x^{2}+y^{2}-4x-2y-11=0$ be a circle. A pair of tangents from the point $\brak{4,5}$ with a pair of radii form a quadrilateral of area

	\hfill\brak{1985-2 Marks}
\item From the origin chords are drawn  to the circle $\brak{x-1}^{2}+y^{2}=1$. The equation of the locus of the mid-points of these chords is

	\hfill\brak{1985-2 Marks}
\item The equation of the line passing through the points of intersection of the circles\\ $3x^{2}+3y^{2}-2x+12y-9=0$ and $x^{2}+y^{2}+6x+2y-15=0$ is

	\hfill\brak{1986-2 Marks}
\item From the point $A\brak{0,3}$ on the circle \\            $x^{2}+4x+\brak{y-3}^{2}=0$, a chord AB is drawn and extended to a point M such that $AM=2AB$. The equation of the locus of M is 

	\hfill\brak{1986-2 Marks}
\item The area of the triangle formed by the tangents from the point $\brak{4,3}$ to the circle $x^{2}+y^{2}=9$ and the line joining their point of contact is

	\hfill\brak{1987-2 Marks}
\item If the circle $C_1:x^{2}+y^{2}=16$ intersects another circle $C_2$ of radius 5 in such a manner that common chord is of maximum length and has a slope equal to $\frac{3}{4}$, then the coordinates of the centre of $C_2$ are

	\hfill\brak{1988-2 Marks}
\item The area formed by the positive x-axis and the normal and the tangent to the circle $x^{2}+y^{2}=4$ at $\brak{1,\sqrt{3}}$ is

	\hfill\brak{1989-2 Marks}
\item If a circle passes through the points of intersection of the coordinate axes with the lines $\lambda x-y+1=0$ and $x-2y+3=0$, then the value of $\lambda =$

	\hfill\brak{1991-2 Marks}
\item The equation of the locus of the mid-points of the circle $4x^{2}+4y^{2}-12x+4y+1=0$ that subtend an angle of $\frac{2\pi}{3}$ at its centre is

	\hfill\brak{1993-2 Marks}
\item The intercept of the line $y=x$ by the circle $x^{2}+y^{2}-2x=0$ is AB. Equation of the circle with AB as a diameter is

	\hfill\brak{1996-1Mark}
\item For each natural number k, let $C_k$ denote the circle with radius k centimetres and centre at the origin. On the circle $C_k$, $\alpha-particle$ moves k centimetres in the counter-clockwise direction. After completing its motion on $C_k$, the particle moves to $C_{k+1}$ in the radial direction. The motion of the particle continues in this manner. The particle starts at $\brak{1,0}$. If the particle crosses the positive direction of the x-axis for the first time on the circle $C_n$ \\then n=

	\hfill\brak{1997-2 Marks}
\item The chords of contact of the pair of tangents drawn from each point on the line $2x+y=4$ to $x^{2}+y^{2}=1$ pass through the point

	\hfill\brak{1997-2 Marks}
\end{enumerate}
\end{document}
}




