\let\negmedspace\undefined
\let\negthickspace\undefined
\documentclass[journal]{IEEEtran}
\usepackage[a5paper, margin=10mm, onecolumn]{geometry}
%\usepackage{lmodern} % Ensure lmodern is loaded for pdflatex
\usepackage{tfrupee} % Include tfrupee package

\setlength{\headheight}{1cm} % Set the height of the header box
\setlength{\headsep}{0mm}  % Set the distance between the header box and the top of the text

\usepackage{gvv-book}
\usepackage{gvv}
\usepackage{cite}
\usepackage{amsmath,amssymb,amsfonts,amsthm}
\usepackage{algorithmic}
\usepackage{graphicx}
\usepackage{textcomp}
\usepackage{xcolor}
\usepackage{txfonts}
\usepackage{listings}
\usepackage{enumitem}
\usepackage{mathtools}
\usepackage{gensymb}
\usepackage{comment}
\usepackage[breaklinks=true]{hyperref}
\usepackage{tkz-euclide} 
\usepackage{listings}
% \usepackage{gvv}                                        
\def\inputGnumericTable{}                                 
\usepackage[latin1]{inputenc}                                
\usepackage{color}                                            
\usepackage{array}                                            
\usepackage{longtable}                                       
\usepackage{calc}                                             
\usepackage{multirow}                                         
\usepackage{hhline}                                           
\usepackage{ifthen}                                           
\usepackage{lscape}
\begin{document}

\bibliographystyle{IEEEtran}
\vspace{3cm}

\title{9.2.18}
\author{EE24BTECH11021 - Eshan Ray}

% \maketitle
% \newpage
% \bigskip
{\let\newpage\relax\maketitle}

\renewcommand{\thefigure}{\theenumi}
\renewcommand{\thetable}{\theenumi}
\setlength{\intextsep}{10pt} % Space between text and floats




\textbf{Question: }\\
Find the area of the region bounded by the curve $y = \sqrt{x}$ and the lines $x = 2y  + 3$ and
the $x-axis$.\\
\solution {
\begin{table}[h!]    
  \centering
  \begin{tabular}[12pt]{ |c| c| c|}
    \hline
	\textbf{Batsman}  & \textbf{Average} & \textbf{Standard Deviation} \\
    \hline
	$K$ &  $31.2$ & $5.21$  \\
    \hline 
	$L$ &  $46.0$ & $6.35$\\
    \hline
	$M$ &  $54.4$ & $6.22$ \\  
    \hline
    	$N$ &  $17.9$ & $5.90$ \\
    \hline         
\end{tabular}

  \caption{Input parameters}
  \label{tab1.1.9.2}
\end{table}
\\
The point of intersection of the line with the parabola is $x_i=h+k_i m$,\\
where,$k_i$ is a constant and is calculated as follows:-
$$k_i=\frac{1}{m^\top Vm}\brak{-m^\top \brak{Vh+u}\pm \sqrt{\sbrak{m^\top \brak{Vh+u}}^2-g\brak{h}\brak{m^\top Vm}}}$$\\
\begin{multline}
     k_i =\frac{1}{\myvec{2&1}\myvec{0&0\\0&1}\myvec{2\\1}}\brak{-\myvec{2&1}\brak{\myvec{0&0\\0&1}\myvec{3\\0}+\myvec{ \frac{-1}{2}\\ 0}}\pm \\
     \sqrt{\sbrak{\myvec{2&1}\brak{\myvec{0&0\\0&1}\myvec{3\\0}+\myvec{ \frac{-1}{2}\\ 0}}}^2-g\brak{h}\brak{\myvec{2&1}\myvec{0&0\\0&1}\myvec{2\\1}}}} 
\end{multline}
We get,\\
$$k_i= -1, 3$$
\begin{align}
     x_1&=\myvec{3\\0}+\brak{-1}\myvec{2\\1}\\
    \implies x_1 &=\myvec{3\\0} + \myvec{-2\\-1}\\
    \implies x_1 &=\myvec{1\\-1}\\
    x_2 &=\myvec{3\\0}+\brak{3}\myvec{2\\1}\\
    \implies x_2&=\myvec{3\\0}+\myvec{6\\3}\\
    \implies x_2&=\myvec{9\\3}
\end{align}
As, y cannot be negative, so the point of intersection of the line and the parabola is $\brak{9,3}$.
$\therefore$ The area bounded by the curve $y = \sqrt{x}$ and line $x = 2y+3$ is given by
\begin{align}
   \int_{0}^{3} \sqrt{x} \,dx+\int_{3}^{9}\sqrt{x}-\brak{\frac{x-3}{2}} \,dx &= \frac{2}{3}\sbrak{x^{\frac{3}{2}}\limits_{0}^{3} \ +\frac{2}{3}\sbrak{x^{\frac{3}{2}}\limits_{3}^{9} \ -\frac{1}{2}\sbrak{\frac{x^2}{2}-3x}\limits_{3}^{9} \  \\
   &= \frac{2}{3}\sbrak{x^{\frac{3}{2}}}\limits_{0}^{9} \  -\frac{1}{2}\brak{\brak{\frac{9^2}{2}-27}-\brak{\frac{3^2}{2}-9}} \\
    &= \frac{2}{3}\brak{\brak{27}-\brak{0}}-\frac{1}{2}\brak{\frac{27}{2}-\frac{-9}{2}}\\
    &=18-9\\
    &= 9 
\end{align}

So, the required area is $9$ units.
   \begin{figure}[!ht]
    \centering
	\includegraphics[width=1\textwidth]{plots/plot.png}
    \caption{Intersection of line and parabola}
    \label{fig:plot}
\end{figure}  

}
\end{document}
