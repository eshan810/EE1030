\let\negmedspace\undefined
\let\negthickspace\undefined
\documentclass[journal,12pt,twocolumn]{IEEEtran}
\usepackage{cite}
\usepackage{amsmath,amssymb,amsfonts,amsthm}
\usepackage{algorithmic}
\usepackage{graphicx}
\usepackage{textcomp}
\usepackage{xcolor}
\usepackage{txfonts}
\usepackage{listings}
\usepackage{enumitem}
\usepackage{mathtools}
\usepackage{gensymb}
\usepackage{comment}
\usepackage[breaklinks=true]{hyperref}
\usepackage{tkz-euclide} 
\usepackage{listings}
\usepackage{gvv}                                        
%\def\inputGnumericTable{}                                 
\usepackage[latin1]{inputenc}                                
\usepackage{color}                                            
\usepackage{array}                                            
\usepackage{longtable}                                       
\usepackage{calc}                                             
\usepackage{multirow}                                         
\usepackage{hhline}                                           
\usepackage{ifthen}                                           
\usepackage{lscape}
\usepackage{tabularx}
\usepackage{array}
\usepackage{float}




\newcommand{\begin{eqnarray}
\newcommand{\EEQA}{\end{eqnarray}
\newcommand{\define}{\stackrel{\triangle}{=}}

\begin{document}
% Marks the beginning of the document

\bibliographystyle{IEEEtran}
\vspace{3cm}

\title{CHAPTER 22- MISCELLANEOUS}
\author{EE24BTECH11021 - Eshan Ray}
\maketitle
\newpage
\bigskip
\renewcommand{\thefigure}{\theenumi}
\renewcommand{\thetable}{\theenumi}


\section*{Section-B \sbrak{JEE Main/AIEEE}}
\begin{enumerate}
    \item[61.] The statement $\sim\brak{p\leftrightarrow \sim q}$ is:     \hfill{\sbrak{JEE-M 2014}}
    \begin{enumerate}
    \item[(a)] a tautology
    \item[(b)] a fallacy
    \item[(c)] equivalent to $p \leftrightarrow q$
    \item[(d)] equivalent to $\sim p \leftrightarrow q$
    \end{enumerate}
    \item[62.] Let A and B be two sets containing four and two sets respectively. Then the number of subsets of $A \times B$, each having at least three elements is: \hfill\sbrak{JEE -M 2015}
    \begin{enumerate}
    \item[(a)] $275$
    \item[(b)] $510$
    \item[(c)] $219$
    \item[(d)] $256$
     \end{enumerate}
    \item[63.] The negation of $\sim s\vee\brak{\sim r\wedge s}$ is equivalent to: \hfill\sbrak{JEE -M 2015}
    \begin{enumerate}
        \item[(a)] $s\vee\brak{r\vee\sim s}$
        \item[(b)] $s\wedge r$
        \item[(c)] $s\wedge \sim r$
        \item[(d)] $s\wedge\brak{r\wedge \sim s}$ 
    \end{enumerate}
    \item[64.] The mean of the data set comprising of $16$ observations is $16$. If one of the observation valued $16$ is deleted and three new observations valued $3, 4$ and $5$ are added to the data, then the mean of the resultant data, is: \hfill\sbrak{JEE -M 2015}
        \begin{enumerate}
            \item[(a)] $15.8$
            \item[(b)] $14.0$
            \item[(c)] $16.8$
            \item[(d)] $16.0$ 
            \end{enumerate}
            \item[65.] If $f\brak{x} + 2f\bigg(\frac{1}{x}\bigg)=3x, x\neq 0$ and\\
            S=$\cbrak{x\mid R: f\brak{x}=f\brak{-x}}$; then S:
             \hfill\sbrak{JEE -M 2016}
            \begin{enumerate}
                \item[(a)] contains exactly two elements.
                \item[(b)] contains more than two elements.
                \item[(c)] is an empty set.
                \item[(d)] contains exactly one element.
            \end{enumerate}
    \item[66.]  The Boolean Expression\\ $\brak{p\wedge\sim q}\vee q\vee\brak{\sim p\wedge q}$ is equivalent to: \hfill\sbrak{JEE -M 2016}
    \begin{enumerate}
        \item[(a)] $p\cup q$
        \item[(b)] $p\vee\sim q$
        \item[(c)] $\sim p\wedge q$
        \item[(d)] $p\cup q$
    \end{enumerate}
\item[67.] If the standard deviation of the numbers $2,3,a$ and $11$ is $3.5$, then which of the following is true? \hfill\sbrak{JEE -M 2016} 
\begin{enumerate}
    \item[(a)] $3a^2-34a+91=0$
    \item[(b)] $3a^2-23a+44=0$
    \item[(c)] $3a^2-26a+55=0$
    \item[(d)] $3a^2-32a+84=0$
\end{enumerate}
    \item[68.] A man is walking towards a vertical pillar in a straight path, at a uniform speed. At a certain point $A$ on the path, he observes that the angle of elevation of the top of the pillar is $30\degree$. After walking for the $10$ minutes from $A$ in the same direction, at a point $B$, he observes that the angle of elevation of the top of the pillar is $60\degree$. Then the time taken \brak {in minutes} by him, from $B$ to reach the pillar, is: \hfill\sbrak{JEE -M 2016}
    \begin{enumerate}
    \item[(a)] $20$
        \item[(b)] $5$
        \item[(c)] $6$
        \item[(d)] $10$ 
    \end{enumerate}
    \item[69.] The following statement\\
    $\brak{p\rightarrow q}\rightarrow\sbrak{\brak{\sim p\rightarrow q}\rightarrow q}$ is:\hfill\sbrak{JEE -M 2017}
    \begin{enumerate}
        \item[(a)] a fallacy
        \item[(b)] a tautology
        \item[(c)] equivalent to $\sim p\rightarrow q$
        \item[(d)] equivalent to $p\rightarrow \sim q$ 
    \end{enumerate}
    \item[70.] $\sum_{i=1}^{9}\brak{x_i-5}=9$ and $\sum_{i=1}^{9}\brak{x_i-5}^2=45$, then the standard deviation of the 9 items $x_1,x_2,.....,x_9$ is:\hfill\sbrak{JEE -M2018}
    \begin{enumerate}
        \item[(a)] $4$
        \item[(b)] $2$
        \item[(c)] $3$
        \item[(d)] $9$ 
    \end{enumerate}
    \item[71.] The Boolean Expression\\
    $\sim\brak{p\vee q}\vee\brak{\sim p\wedge q}$ is equivalent to: \hfill\sbrak{JEE -M 2018}
    \begin{enumerate}
        \item[(a)] p
        \item[(b)] q
        \item[(c)] $\sim q$
        \item[(d)] $\sim p$ 
    \end{enumerate}
    \item[72.] Let S=$\cbrak{x\in R:x\geq 0}$ and\\
$2\left|\sqrt{x}-3\right|+\sqrt{x}\brak{\sqrt{x}-6}+6=0$. Then S:\hfill\sbrak{JEE-M 2018}
\begin{enumerate}
    \item[(a)] contains exactly one element.
    \item[(b)] contains exactly two elements.
    \item[(c)] contains exactly four elements.
    \item[(d)] is an empty set. 
\end{enumerate}
\item[73.] If the Boolean expression\\
$\brak{p\oplus q}\wedge\brak{\sim p\odot q}$ is equivalent to\\
$p\wedge q$, where $\oplus,\odot\in \cbrak{\wedge,\vee}$ then the ordered pair $\brak{\oplus,\odot}$ is:
\hfill\sbrak{JEE -M 2019-9 JAN}
\begin{enumerate}
    \item[(a)] $\brak{\vee,\wedge}$
    \item[(b)] $\brak{\vee,\vee}$
    \item[(c)] $\brak{\wedge,\vee}$
    \item[(d)] $\brak{\wedge,\wedge}$ 
\end{enumerate}
\item[74.]  $5$ students of a class have an average height $150$ cm and variance $18$ $cm^2$. A new student, whose height is $156$ cm joined them. The variance$\brak{in cm^2}$ of the height of these six students is:\hfill\sbrak{JEE -M 2019-9 JAN} 
\begin{enumerate}
    \item[(a)] $16$
    \item[(b)] $22$
    \item[(c)] $20$
    \item[(d)] $18$
\end{enumerate}
\item[75.] If the standard deviation of the numbers $-1,0,1,k$ is $\sqrt{5}$ where $k>0$, then $k$ is equal to:\hfill\sbrak{JEE -M 2019-9 April}
\begin{enumerate}
    \item[(a)] $2\sqrt{6}$
    \item[(b)] $2\sqrt{\frac{10}{3}}$
    \item[(c)] $4\sqrt{\frac{5}{3}}$
    \item[(d)] $\sqrt{6}$ 
\end{enumerate}
\item[76.] For any two statements p and q, the negative of the expression $p\vee\brak{\sim p\wedge q}$ is:\hfill\sbrak{JEEM2019-9 April}
\begin{enumerate}
    \item[(a)] $\sim p\wedge\sim q$
    \item[(b)] $p\wedge q$
    \item[(c)] $p\leftrightarrow q$
    \item[(d)] $\sim p\vee\sim q$ 
\end{enumerate}
    
\end{enumerate}
\end{document}
