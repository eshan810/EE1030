\let\negmedspace\undefined
\let\negthickspace\undefined
\documentclass[journal]{IEEEtran}
\usepackage[a5paper, margin=10mm, onecolumn]{geometry}
%\usepackage{lmodern} % Ensure lmodern is loaded for pdflatex
\usepackage{tfrupee} % Include tfrupee package

\setlength{\headheight}{1cm} % Set the height of the header box
\setlength{\headsep}{0mm}  % Set the distance between the header box and the top of the text

\usepackage{gvv-book}
\usepackage{gvv}
\usepackage{cite}
\usepackage{amsmath,amssymb,amsfonts,amsthm}
\usepackage{algorithmic}
\usepackage{graphicx}
\usepackage{textcomp}
\usepackage{xcolor}
\usepackage{txfonts}
\usepackage{listings}
\usepackage{enumitem}
\usepackage{mathtools}
\usepackage{gensymb}
\usepackage{comment}
\usepackage[breaklinks=true]{hyperref}
\usepackage{tkz-euclide} 
\usepackage{listings}
% \usepackage{gvv}                                        
\def\inputGnumericTable{}                                 
\usepackage[latin1]{inputenc}                                
\usepackage{color}                                            
\usepackage{array}                                            
\usepackage{longtable}                                       
\usepackage{calc}                                             
\usepackage{multirow}                                         
\usepackage{hhline}                                           
\usepackage{ifthen}                                           
\usepackage{lscape}
\begin{document}

\bibliographystyle{IEEEtran}
\vspace{3cm}

\title{29/01/2024-Shift 1}
\author{EE24BTECH11021 - Eshan Ray}

% \maketitle
% \newpage
% \bigskip
{\let\newpage\relax\maketitle}

\renewcommand{\thefigure}{\theenumi}
\renewcommand{\thetable}{\theenumi}
\setlength{\intextsep}{10pt} % Space between text and floats

\begin{enumerate}
\setcounter{enumi}{15}
    \item Let $PQR$ be a triangle with $R\brak{-1,4,2}$. Suppose $M\brak{2,1,2}$ is the mid-point of $PQ$. The distance of the centroid of $\triangle PQR$ from the point of intersection of the lines $\frac{x-2}{0}=\frac{y}{2}=\frac{z+3}{-1}$ and $\frac{x-1}{1}=\frac{y+3}{-3}=\frac{z+1}{1}$ is
        \begin{enumerate}
            \item $\sqrt{99}$
            \item $9$
            \item $\sqrt{69}$
            \item $69$
        \end{enumerate}
    \item Let $\overrightarrow{a},\overrightarrow{b}$ and $\overrightarrow{c}$ be three non-zero vectors such that $\overrightarrow{b}$ and $\overrightarrow{c}$ are non-collinear. If $\overrightarrow{a}+5\overrightarrow{b}$ is collinear with $\overrightarrow{c},\overrightarrow{b}+6\overrightarrow{c}$ is collinear with $\overrightarrow{a}$ and $\overrightarrow{a}+\alpha\overrightarrow{b}+\beta\overrightarrow{c}=\overrightarrow{0},$ then $\alpha+\beta$ is equal to 
        \begin{enumerate}
            \item $-25$
            \item $35$
            \item $-30$
            \item $30$
        \end{enumerate}
    \item If $z=\frac{1}{2}-2i$ is such that $\abs{z+1}=\alpha z+\beta\brak{1+i},i=\sqrt{-1}$ and $\alpha,\beta\in R$, then $\alpha+\beta$ is equal to 
        \begin{enumerate}
            \item $-1$
            \item $-4$
            \item $2$
            \item $3$
        \end{enumerate}
    \item Let $O$ be the origin and the position vectors of $A$ and $B$ be $2\hat{i}+2\hat{j}+\hat{k}$ and $2\hat{i}+4\hat{j}+4\hat{k}$ respectively. If the internal bisector of $\angle AOB$ meets the line $AB$ at $C$, then the length of $OC$ is
        \begin{enumerate}
            \item $\frac{3}{2}\sqrt{34}$
            \item $\frac{3}{2}\sqrt{31}$
            \item $\frac{2}{3}\sqrt{34}$
            \item $\frac{2}{3}\sqrt{31}$
        \end{enumerate}
    \item If the value of the integral $\int_{-\frac{\pi}{2}}^{\frac{\pi}{2}}\brak{\frac{x^2\cos{x}}{1+\pi^x}+\frac{1+\sin^2{x}}{1+e^{\sin{x}^{2023}}}} \, dx=\frac{\pi}{4}\brak{\pi+a}-2$, then the value of $a$ is 
        \begin{enumerate}
            \item $2$
            \item $-\frac{3}{2}$
            \item $\frac{3}{2}$
            \item $3$
        \end{enumerate}
    \item A line with direction ratio $2,1,2$ meets the lines $x=y+2=z$ and $x+2=2y=2z$ respectively at points $P$ and $Q$. If the length of the perpendicular from the point $\brak{1,2,12}$ to the line $PQ$ is $l,$ then $l^2$ is \dots
    \item The area \brak{in\,sq.\,units} of the part of the circle $x^2+y^2=169$ which is below the line $5x-y=13$ is $\frac{\pi\alpha}{2\beta}-\frac{65}{2}+\frac{\alpha}{\beta}\sin^{-1}\brak{\frac{12}{13}}$, where $\alpha,\beta$ are coprime numbers. Then $\alpha+\beta$ is equal to \dots
    \item If the solution curve $y=y\brak{x}$ to the differential equation $\brak{1+y^2}\brak{+\log_{e}x}dx+x dy=0,x\textgreater 0$ passes through the point $\brak{1,1}$ and $y\brak{e}=\frac{\alpha-\tan\brak{\frac{3}{2}}}{\beta+\tan\brak{\frac{3}{2}}}$, then $\alpha+2\beta$ is \dots
    \item If the mean and variance of the data $65, 68, 58, 44, 48, 45, 60, \alpha, \beta, 60 where \alpha > \beta$ are $56$ and $66.2$ respectively, then $\alpha^2+\beta^2$ is equal to\dots
    \item If $\frac{\binom{11}{1}}{2}+\frac{\binom{11}{2}}{3}+\dots+\frac{\binom{11}{9}}{10}=\frac{n}{m}$with $gcd\brak{m,n}=1,$ then $m+n$ is equal to \dots 
    \item If the points of intersection of two conics $x^2+y^2=4b$ and $\frac{x^2}{16}+\frac{y^2}{b^2}=1$ lie on the curve $y^2=3x^2$, then $3\sqrt{3}$ times the area of the rectangle formed by the intersection points is \dots
    \item Let $\alpha,\beta$ be the roots of the equation $x^2-x+2=0$ with $Im\brak{\alpha}\textgreater Im\brak{\beta}$. Then $\alpha^6+\alpha^4+\beta^4-5\alpha^2$ is equal to \dots
    \item Equations of two diameters of a circle are $2x-3y=5$ and $3x-4y=7$. The line joining the points $\brak{-\frac{22}{7},-4}$ and $\brak{-\frac{1}{7},3}$ intersects the circle at only one point $P\brak{\alpha,\beta}$. Then, $17\beta-\alpha$ is equal to \dots
    \item All the letters of the word $"GTWENTY"$ are written in all possible ways with or without meaning and these words are written as in a dictionary. The serial number of the word $"GTWENTY"$ is \dots
    \item Let $f\brak{x}=2^x-x^2,x\in R$. If $m$ and $n$ are respectively the number of points t which the curves $y=f\brak{x}$ and $y=f\prime\brak{x}$ intersect the $x-axis$ then the value of $m+n$ is \dots
\end{enumerate}
\end{document}
