\let\negmedspace\undefined
\let\negthickspace\undefined
\documentclass[journal]{IEEEtran}
\usepackage[a5paper, margin=10mm, onecolumn]{geometry}
%\usepackage{lmodern} % Ensure lmodern is loaded for pdflatex
\usepackage{tfrupee} % Include tfrupee package

\setlength{\headheight}{1cm} % Set the height of the header box
\setlength{\headsep}{0mm}  % Set the distance between the header box and the top of the text

\usepackage{gvv-book}
\usepackage{gvv}
\usepackage{cite}
\usepackage{amsmath,amssymb,amsfonts,amsthm}
\usepackage{algorithmic}
\usepackage{graphicx}
\usepackage{textcomp}
\usepackage{xcolor}
\usepackage{txfonts}
\usepackage{listings}
\usepackage{enumitem}
\usepackage{mathtools}
\usepackage{gensymb}
\usepackage{comment}
\usepackage[breaklinks=true]{hyperref}
\usepackage{tkz-euclide} 
\usepackage{listings}
% \usepackage{gvv}                                        
\def\inputGnumericTable{}                                 
\usepackage[latin1]{inputenc}                                
\usepackage{color}                                            
\usepackage{array}                                            
\usepackage{longtable}                                       
\usepackage{calc}                                             
\usepackage{multirow}                                         
\usepackage{hhline}                                           
\usepackage{ifthen}                                           
\usepackage{lscape}
\begin{document}

\bibliographystyle{IEEEtran}
\vspace{3cm}

\title{03/09/2020-Shift 2}
\author{EE24BTECH11021 - Eshan Ray}

% \maketitle
% \newpage
% \bigskip
{\let\newpage\relax\maketitle}

\renewcommand{\thefigure}{\theenumi}
\renewcommand{\thetable}{\theenumi}
\setlength{\intextsep}{10pt} % Space between text and floats

\begin{enumerate}
\setcounter{enumi}{15}
    \item If $\int \sin^{-1}\brak{{\sqrt{\frac{x}{1+x}}}} \,dx=A\brak{x}\tan^{-1}\brak{{\sqrt{x}}}+B\brak{x}+C$, where $C$ is a constant of integration, then the ordered pair $\brak{A\brak{x},B\brak{x}}$ can be$\colon$
        \begin{enumerate}
            \item $\brak{x+1,-\sqrt{x}}$
            \item $\brak{x-1,-\sqrt{x}}$
            \item $\brak{x+1,\sqrt{x}}$
            \item $\brak{x-1,\sqrt{x}}$
        \end{enumerate}
    \item If the sum of the series $20+19\frac{3}{5}+19\frac{1}{5}+18\frac{4}{5}+\dots$ upto $n^{th}$ term is $488$ and the $n^{th}$ term is negative, then$\colon$
        \begin{enumerate}
            \item $n=60$
            \item $n=41$
            \item $n^{th}$ term is $-4$
            \item $n^{th}$ term is $-4\frac{2}{5}$
        \end{enumerate}
    \item Let $p,q,r$ be three statements such that the truth value of $\brak{p\wedge q}\rightarrow\brak{\sim p\vee r}$ is $F$. The truth values of $p,q,r$ are respectively$\colon$
        \begin{enumerate}
            \item $F,T,F$
            \item $T,F,T$
            \item $T,T,F$
            \item $T,T,T$
        \end{enumerate}
    \item If the surface area of the cube is increasing at a rate of $3.6 cm^2/sec$,retaining its shape; then the rate of change of volume $\brak{in cm^2/sec}$, when the length of the cube is $10 cm$ is$\colon$
        \begin{enumerate}
            \item $9$
            \item $10$
            \item $18$
            \item $20$
        \end{enumerate}
    \item   Let $R_1$ and $R_2$ be two relations defined as follows$\colon$\\
            $R_1=\cbrak{\brak{a,b}\in R^2\colon a^2+b^2 \in Q}$ and\\
            $R_2=\cbrak{\brak{a,b}\in R^2\colon a^2+b^2 \notin Q}$, where $Q$ is set of all rational numbers. Then$\colon$
        \begin{enumerate}
            \item $R_1$ is transitive but $R_2$ is not transitive
            \item $R_1$ and $R_2$  are both transitive
            \item $R_2$ is transitive but $R_1$ is not transitive
            \item Neither $R_1$ nor $R_2$ is transitive
        \end{enumerate}
    \item If $m$ arithmetic means\brak{A.Ms} and three geometric means \brak{G.Ms} are inserted between $3$ and $243$ such that the $4^{th} A.M.$ is equal to $2^{nd}G.M.$, then $m$ is equal to\dots
    \item Let a plane $P$ contain two lines $\overrightarrow{r}=\hat{i}+\lambda\brak{\hat{i}+\hat{j}},\lambda\in R$ and $\overrightarrow{r}= -\hat{j}+\mu\brak{\hat{j}-\hat{k}},\mu\in R$. If $Q\brak{\alpha,\beta,\gamma}$ is the foot of the perpendicular drawn from the point $M\brak{1,0,1}$ to $P$, then $3\brak{\alpha,\beta,\gamma}$ equals\dots
    \item Let $S$ be set of all integer solutions $\brak{x,y,z}$, of the system of equations\\
        $x-2y+5z=0$\\
        $-2x+4y+z=0$\\
        $-7x+14y+9z=0$\\
        such that $15\leq x^2+y^2+z^2\leq 150$. Then the number of elements in the set $S$ is equal to \dots
    \item The total number of $3-$digit number numbers, whose sum of digits is $10$, is\dots
    \item If the tangent at the curve, $y=e^x$ at a point $\brak{c,e^c}$ and the normal to the parabola, $y^2=4x$ at the point $\brak{1,2}$ intersect at the same point on the $x-axis$, then the value of $c$ is\dots 
    
\end{enumerate}
\end{document}
