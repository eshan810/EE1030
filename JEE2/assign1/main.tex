\let\negmedspace\undefined
\let\negthickspace\undefined
\documentclass[journal]{IEEEtran}
\usepackage[a5paper, margin=10mm, onecolumn]{geometry}
%\usepackage{lmodern} % Ensure lmodern is loaded for pdflatex
\usepackage{tfrupee} % Include tfrupee package

\setlength{\headheight}{1cm} % Set the height of the header box
\setlength{\headsep}{0mm}  % Set the distance between the header box and the top of the text

\usepackage{gvv-book}
\usepackage{gvv}
\usepackage{cite}
\usepackage{amsmath,amssymb,amsfonts,amsthm}
\usepackage{algorithmic}
\usepackage{graphicx}
\usepackage{textcomp}
\usepackage{xcolor}
\usepackage{txfonts}
\usepackage{listings}
\usepackage{enumitem}
\usepackage{mathtools}
\usepackage{gensymb}
\usepackage{comment}
\usepackage[breaklinks=true]{hyperref}
\usepackage{tkz-euclide} 
\usepackage{listings}
% \usepackage{gvv}                                        
\def\inputGnumericTable{}                                 
\usepackage[latin1]{inputenc}                                
\usepackage{color}                                            
\usepackage{array}                                            
\usepackage{longtable}                                       
\usepackage{calc}                                             
\usepackage{multirow}                                         
\usepackage{hhline}                                           
\usepackage{ifthen}                                           
\usepackage{lscape}
\begin{document}

\bibliographystyle{IEEEtran}
\vspace{3cm}

\title{03/09/2020-Shift 2}
\author{EE24BTECH11021 - Eshan Ray}

% \maketitle
% \newpage
% \bigskip
{\let\newpage\relax\maketitle}

\renewcommand{\thefigure}{\theenumi}
\renewcommand{\thetable}{\theenumi}
\setlength{\intextsep}{10pt} % Space between text and floats

\begin{enumerate}
\setcounter{enumi}{15}
    \item If $x^3dy+xy dx=x^2dy+2y dx;y\brak{2}=e$ and $x\textgreater 1$, then $y\brak{4}$ is equal to $\colon$
    \hfill{[Sep-2020]}
        \begin{enumerate}
            \item $\frac{3}{2}+\sqrt{e}$
            \item $\frac{3}{2}\sqrt{e}$
            \item $\frac{1}{2}+\sqrt{e}$
            \item $\frac{\sqrt{e}}{2}$
        \end{enumerate}
    \item Let $e_1$ and $e_2$ be eccentricities of the ellipse,$\frac{x^2}{25}+\frac{y^2}{b^2}=1\brak{b\textless 5}$ and the hyperbola, $\frac{x^2}{16}-\frac{y^2}{b^2}=1$ respectively satisfying $e_1e_2=1$. If $\alpha$ and $\beta$ are the distances between the foci of the ellipse and the foci of the hyperbola respectively, then the ordered pair $\brak{\alpha,\beta}$ is equal to $\colon$
    \hfill{[Sep-2020]}
        \begin{enumerate}
            \item $\brak{8,10}$
            \item $\brak{8,12}$
            \item $\brak{\frac{20}{3},12}$
            \item $\brak{\frac{24}{5},10}$
        \end{enumerate}
    \item The set of all real values of $\lambda$ for which the quadratic equations,\\ $\brak{\lambda^2+1}x^2-4\lambda x+2=0$ always has exactly one root in the interval $\brak{0,1}$ is $\colon$
    \hfill{[Sep-2020]}
        \begin{enumerate}
            \item $\brak{-3,-1}$
            \item $\left(1,3 \right]$
            \item $\brak{0,2}$
            \item $\left(2,4 \right]$
        \end{enumerate}
    \item If the term independent of $x$ in the expansion of $\brak{\frac{3}{2}x^2-\frac{1}{3x}}^9$ is $k$, then $18k$ is equal to $\colon$
    \hfill{[Sep-2020]}
        \begin{enumerate}
            \item $9$
            \item $11$
            \item $5$
            \item $7$
        \end{enumerate}
    \item Let $p,q,r$ be three statements such that the truth value of $\brak{p\wedge q}\rightarrow\brak{\sim p\vee r}$ is $F$. The truth values of $p,q,r$ are respectively$\colon$
    \hfill{[Sep-2020]}
        \begin{enumerate}
            \item $F,T,F$
            \item $T,F,T$
            \item $T,T,F$
            \item $T,T,T$
        \end{enumerate}
    \item If $m$ arithmetic means\brak{A.Ms} and three geometric means \brak{G.Ms} are inserted between $3$ and $243$ such that the $4^{th} A.M.$ is equal to $2^{nd}G.M.$, then $m$ is equal to\dots
    \hfill{[Sep-2020]}
    \item Let a plane $P$ contain two lines $\overrightarrow{r}=\hat{i}+\lambda\brak{\hat{i}+\hat{j}},\lambda\in R$ and $\overrightarrow{r}= -\hat{j}+\mu\brak{\hat{j}-\hat{k}},\mu\in R$. If $Q\brak{\alpha,\beta,\gamma}$ is the foot of the perpendicular drawn from the point $M\brak{1,0,1}$ to $P$, then $3\brak{\alpha,\beta,\gamma}$ equals\dots
    \hfill{[Sep-2020]}
    \item Let $S$ be set of all integer solutions $\brak{x,y,z}$, of the system of equations\\
        $x-2y+5z=0$\\
        $-2x+4y+z=0$\\
        $-7x+14y+9z=0$\\
        such that $15\leq x^2+y^2+z^2\leq 150$. Then the number of elements in the set $S$ is equal to \dots
       \hfill{[Sep-2020]}
    \item The total number of $3-$digit number numbers, whose sum of digits is $10$, is\dots
    \hfill{[Sep-2020]}
    \item If the tangent at the curve, $y=e^x$ at a point $\brak{c,e^c}$ and the normal to the parabola, $y^2=4x$ at the point $\brak{1,2}$ intersect at the same point on the $x-axis$, then the value of $c$ is\dots 
    \hfill{[Sep-2020]}
    
\end{enumerate}
\end{document}
