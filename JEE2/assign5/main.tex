\let\negmedspace\undefined
\let\negthickspace\undefined
\documentclass[journal]{IEEEtran}
\usepackage[a5paper, margin=10mm, onecolumn]{geometry}
%\usepackage{lmodern} % Ensure lmodern is loaded for pdflatex
\usepackage{tfrupee} % Include tfrupee package

\setlength{\headheight}{1cm} % Set the height of the header box
\setlength{\headsep}{0mm}  % Set the distance between the header box and the top of the text

\usepackage{gvv-book}
\usepackage{gvv}
\usepackage{cite}
\usepackage{amsmath,amssymb,amsfonts,amsthm}
\usepackage{algorithmic}
\usepackage{graphicx}
\usepackage{textcomp}
\usepackage{xcolor}
\usepackage{txfonts}
\usepackage{listings}
\usepackage{enumitem}
\usepackage{mathtools}
\usepackage{gensymb}
\usepackage{comment}
\usepackage[breaklinks=true]{hyperref}
\usepackage{tkz-euclide} 
\usepackage{listings}
% \usepackage{gvv}                                        
\def\inputGnumericTable{}                                 
\usepackage[latin1]{inputenc}                                
\usepackage{color}                                            
\usepackage{array}                                            
\usepackage{longtable}                                       
\usepackage{calc}                                             
\usepackage{multirow}                                         
\usepackage{hhline}                                           
\usepackage{ifthen}                                           
\usepackage{lscape}
\begin{document}

\bibliographystyle{IEEEtran}
\vspace{3cm}

\title{18/03/2021-Shift 1}
\author{EE24BTECH11021 - Eshan Ray}

% \maketitle
% \newpage
% \bigskip
{\let\newpage\relax\maketitle}

\renewcommand{\thefigure}{\theenumi}
\renewcommand{\thetable}{\theenumi}
\setlength{\intextsep}{10pt} % Space between text and floats

\begin{enumerate}
    \item If the functions are defined as $f\brak{x}=\sqrt{x}$ and $g\brak{x}=\sqrt{1-x}$, then what is the common domain of the following functions$\colon f+g,f-g,f/g,g/f,g-f$ where $\brak{f\pm g}\brak{x}=f\brak{x}=f\brak{x}\pm g\brak{x},\brak{f/g}\brak{x}=\frac{f\brak{x}}{g\brak{x}}$
        \begin{enumerate}
            \item $0\textless x\leq 1$
            \item $0\leq x\textless 1$
            \item $0\leq x\leq 1$
            \item $0\textless x\textless 1$
        \end{enumerate}
    \item Let $\alpha,\beta,\gamma$ be the roots of the equation, $x^3+ax^2+bx+c=0,\brak{a,b,c\in R\, and\, a,b\, and\, a,b\neq 0}$. The system of equations $\brak{in\, u,v,w}$ given by $\alpha u+\beta v+\gamma w=0;\beta u+\gamma v+\alpha w=0;\gamma u+\alpha v+\beta w=0$ has non-trivial solutions,then the value of $\frac{a^2}{b}$ is
        \begin{enumerate}
            \item $5$
            \item $1$
            \item $0$
            \item $3$
        \end{enumerate}
    \item If the equation $a\abs{z}^2+\overline{\Bar{\alpha}z+\alpha \Bar{z}}+d=0$ represents a circle where $a,d$ are real constants then which of the following conditions are correct?
        \begin{enumerate}
            \item $\abs{\alpha}^2-ad\neq 0$
            \item $\abs{\alpha}^2-ad\textgreater 0$ and $a\in R-\cbrak{0}$
            \item $\alpha=0,a,d\in R^{+}$
            \item $\abs{\alpha}^2-ad\geq 0$ and $a\in R$
        \end{enumerate}
    \item $\frac{1}{3^2-1}+\frac{1}{5^2-1}+\frac{1}{7^2-1}+\dots+\frac{1}{201^2-1}$ is equal to $\colon$  
        \begin{enumerate}
            \item $\frac{101}{404}$
            \item $\frac{101}{408}$
            \item $\frac{99}{400}$
            \item $\frac{25}{101}$
        \end{enumerate}
    \item The number of integral values of $m$ so that the abscissa of point of intersection of lines $3x + 4y = 9$ and $y = mx + 1$ is also an integer,           is$\colon$
        \begin{enumerate}
            \item $3$
            \item $2$
            \item $1$
            \item $0$
        \end{enumerate}
    \item The solutions of the equation
        \begin{align*}
            \mydet{
                    1+\sin^2{x} & \sin^2{x} & \sin^2{x}\\
                    \cos^2{x} & 1+\cos^2{x} & \cos^2{x}\\
                    4\sin{2x} & 4\sin{2x} & 1+4\sin{2x}\\
            }
            =0
        \end{align*},
        $\brak{0\textless x\textless \pi}$, are$\colon$
        \begin{enumerate}
            \item $\frac{\pi}{6},\frac{5\pi}{6}$
            \item $\frac{7\pi}{12},\frac{11\pi}{12}$
            \item $\frac{5\pi}{12},\frac{7\pi}{12}$
            \item $\frac{\pi}{12},\frac{\pi}{6}$
        \end{enumerate}
    \item If $f\brak{x}=
            \begin{cases}
                \frac{1}{\abs{x}} & \text{;} \abs{x}\geq 1\\
                ax^2+b & \text{;} \abs{x}\textless 1
            \end{cases}
    $
        is differentiable at every point of the domain,then the values of $a$ and $b$ are respectively $\colon$
        \begin{enumerate}
            \item $\frac{5}{2},-\frac{3}{2}$
            \item $-\frac{1}{2},\frac{3}{2}$
            \item $\frac{1}{2},\frac{1}{2}$
            \item $\frac{1}{2},-\frac{3}{2}$
        \end{enumerate}
    \item A vector $\vec{a}$ has components $3p$ and $1$ with respect to a rectangular Cartesian system. This system is rotated through a certain angle about the origin in the counterclockwise sense. If with respect to the new system, $\vec{a}$ has components $p + 1$ and $\sqrt{10}$, then a value of $p$ is equal to$\colon$
        \begin{enumerate}
            \item $1$
            \item $-1$
            \item $\frac{4}{5}$
            \item $-\frac{5}{4}$
        \end{enumerate}
    \item The sum of all the $4-digit$ distinct numbers that can be formed with the digits $1, 2, 2$ and $3$ is$\colon$
        \begin{enumerate}
            \item $26664$
            \item $122664$
            \item $122234$
            \item $22264$
        \end{enumerate}
    \item Choose the correct statement about two circles whose equations are given below$\colon$\\
    $x^2 + y^2-10x-10y+41=0$\\
    $x^2 + y^2-22x-10y+137=0$
        \begin{enumerate}
            \item circles have no meeting point
            \item circles have two meeting points
            \item circles have only one meeting point
            \item circles have the same centre
        \end{enumerate}
    \item If $\alpha,\beta$ are natural numbers such that $100^\alpha-199\beta=\brak{100}\brak{100}+\brak{99}\brak{101}+\brak{98}\brak{102}+\dots+\brak{1}\brak{199}$, then the slope of the line passing through $\brak{\alpha,\beta}$ and origin is$\colon$
        \begin{enumerate}
            \item $510$
            \item $550$
            \item $540$
            \item $530$
        \end{enumerate}
    \item The value of $3+\frac{1}{4+\frac{1}{3+\frac{1}{4+\frac{1}{3+\dots\infty}}}}$ is equal to $\colon$
        \begin{enumerate}
            \item $3+2\sqrt{3}$
            \item $4+\sqrt{3}$
            \item $2+\sqrt{3}$
            \item $1.5+\sqrt{3}$
        \end{enumerate}
    \item The integral $\int \frac{\brak{2x-1}\cos{\sqrt{\brak{2x-1}^2+5}}}{\sqrt{4x^2-4x+6}} \, dx$ is equal to \brak{where\, c\, is\,a\,constant\,of\,integration}
        \begin{enumerate}
            \item $\frac{1}{2}\sin{\sqrt{\brak{2x+1}^2+5}}+c$
            \item $\frac{1}{2}\sin{\sqrt{\brak{2x-1}^2+5}}+c$
            \item $\frac{1}{2}\cos{\sqrt{\brak{2x+1}^2+5}}+c$
            \item $\frac{1}{2}\cos{\sqrt{\brak{2x-1}^2+5}}+c$
        \end{enumerate}
    \item The differential equations satisfied by the system of parabolas $y^2=4a\brak{x+a}$ is $\colon$
        \begin{enumerate}
            \item $y\brak{\frac{dy}{dx}}+2x\brak{\frac{dy}{dx}}-y=0$
            \item $y\brak{\frac{dy}{dx}}^2+2x\brak{\frac{dy}{dx}}-y=0$
            \item $y\brak{\frac{dy}{dx}}^2-2x\brak{\frac{dy}{dx}}-y=0$
            \item $y\brak{\frac{dy}{dx}}^2-2x\brak{\frac{dy}{dx}}+y=0$
        \end{enumerate}
    \item The real-valued function $f\brak{x}=\frac{cosec^{-1}{x}}{\sqrt{x-\sbrak{x}}}$, where $\sbrak{x}$ denotes the greatest integer less than or equal to $x$, is defined for all $x$ belonging to $\colon$
        \begin{enumerate}
            \item all non- integers except the interval $\sbrak{-1,1}$
            \item all integers except $0, -1, 1$
            \item all reals except integers
            \item all reals except the interval $\sbrak{-1,1}$
        \end{enumerate}
        
\end{enumerate}
\end{document}
