\let\negmedspace\undefined
\let\negthickspace\undefined
\documentclass[journal]{IEEEtran}
\usepackage[a5paper, margin=10mm, onecolumn]{geometry}
%\usepackage{lmodern} % Ensure lmodern is loaded for pdflatex
\usepackage{tfrupee} % Include tfrupee package

\setlength{\headheight}{1cm} % Set the height of the header box
\setlength{\headsep}{0mm}  % Set the distance between the header box and the top of the text

\usepackage{gvv-book}
\usepackage{gvv}
\usepackage{cite}
\usepackage{amsmath,amssymb,amsfonts,amsthm}
\usepackage{algorithmic}
\usepackage{graphicx}
\usepackage{textcomp}
\usepackage{xcolor}
\usepackage{txfonts}
\usepackage{listings}
\usepackage{enumitem}
\usepackage{mathtools}
\usepackage{gensymb}
\usepackage{comment}
\usepackage[breaklinks=true]{hyperref}
\usepackage{tkz-euclide} 
\usepackage{listings}
% \usepackage{gvv}                                        
\def\inputGnumericTable{}                                 
\usepackage[latin1]{inputenc}                                
\usepackage{color}                                            
\usepackage{array}                                            
\usepackage{longtable}                                       
\usepackage{calc}                                             
\usepackage{multirow}                                         
\usepackage{hhline}                                           
\usepackage{ifthen}                                           
\usepackage{lscape}
\begin{document}

\bibliographystyle{IEEEtran}
\vspace{3cm}

\title{03/09/2020-Shift 1}
\author{EE24BTECH11021 - Eshan Ray}

% \maketitle
% \newpage
% \bigskip
{\let\newpage\relax\maketitle}

\renewcommand{\thefigure}{\theenumi}
\renewcommand{\thetable}{\theenumi}
\setlength{\intextsep}{10pt} % Space between text and floats

\begin{enumerate}
\setcounter{enumi}{15}
    \item If the number of integral terms in the expansion of $\brak{3^{\frac{1}{2}}+5^{\frac{1}{8}}}^n$ is exactly $33$, then the least value of n is $\colon$
    \hfill{[Sep-2020]}
        \begin{enumerate}
            \item $264$
            \item $256$
            \item $128$
            \item $248$
        \end{enumerate}
    \item If $\alpha$ and $\beta$ are the roots of the equation $x^2+px+2=0$ and $\frac{1}{\alpha}$ and $\frac{1}{\beta}$ are the roots of the equation $2x^2+2qx+1=0$, then $\brak{\alpha-\frac{1}{\alpha}}\brak{\beta-\frac{1}{\beta}}\brak{\alpha + \frac{1}{\beta}}\brak{\beta +\frac{1}{\alpha}}$ is equal to$\colon$
    \hfill{[Sep-2020]}
        \begin{enumerate}
            \item $\frac{9}{4}\brak{9+p^2}$
            \item $\frac{9}{4}\brak{9+q^2}$
            \item $\frac{9}{4}\brak{9-p^2}$
            \item $\frac{9}{4}\brak{9-q^2}$
        \end{enumerate}
    \item Let $\sbrak{t}$ denote the greatest integer $\leq t$. If for some $\lambda\in R-\cbrak{0,1}$,\\
    $\lim_{x \to 0} \abs{\frac{1-x+\abs{x}}{\lambda-x+\sbrak{x}}}=L$, then $L$ is equal to$\colon$
   \hfill{[Sep-2020]}
        \begin{enumerate}
            \item $0$
            \item $2$
            \item $\frac{1}{2}$
            \item $1$
        \end{enumerate}
    \item $2\pi -\brak{\sin^{-1}{\frac{4}{5}}+\sin^{-1}{\frac{5}{13}}+\sin^{-1}{\frac{16}{65}}}$ is equal to $\colon$
    \hfill{[Sep-2020]}
        \begin{enumerate}
            \item $\frac{7\pi}{4}$
            \item $\frac{5\pi}{4}$
            \item $\frac{3\pi}{2}$
            \item $\frac{\pi}{2}$
        \end{enumerate}
    \item The proposition $p \rightarrow \sim \brak{p\wedge \sim q}$ is equivalent to $\colon$
    \hfill{[Sep-2020]}
        \begin{enumerate}
            \item $\brak{\sim p}\vee q$
            \item $q$
            \item $\brak{\sim p}\wedge q$
            \item $\brak{\sim p}\vee \brak{\sim q}$
        \end{enumerate}
    \item If $\lim_{x \to 0} \cbrak{\frac{1}{x^8}\brak{1-\cos{\frac{x^2}{2}}-\cos{\frac{x^2}{4}}+\cos{\frac{x^2}{2}}\cos{\frac{x^2}{4}}}}=2^{-k}$, then the value of $k$ is \dots 
    \hfill{[Sep-2020]}
    \item The diameter of the circle, whose centre lies on the line $x + y = 2$ in the first quadrant and which touches both the lines $x=3$ and $y=2$, is\dots
    \hfill{[Sep-2020]}
    \item The value of $\brak{0.16}^{\log_{2.5}\brak{\frac{1}{3}+\frac{1}{3^2}+\frac{1}{3^3}\dots to \infty}}$ is equal to \dots 
    \hfill{[Sep-2020]}
    \item Let $A=\myvec{x &1 \\1 &0},x\in R$ and $A^4=\sbrak{a_{ij}}$. If $a_{11}=109$, then $a_{22}$ is equal to \dots
    \hfill{[Sep-2020]}
    \item If $\brak{\frac{1+i}{1-i}}^{m/2}=\brak{\frac{1+i}{i-1}}^{n/3}=1,\brak{m,n\in N}$ then the greatest common divisor of the least values of $m$ and $n$ is \dots
    \hfill{[Sep-2020]}
    
\end{enumerate}
\end{document}
