\let\negmedspace\undefined
\let\negthickspace\undefined
\documentclass[journal]{IEEEtran}
\usepackage[a5paper, margin=10mm, onecolumn]{geometry}
%\usepackage{lmodern} % Ensure lmodern is loaded for pdflatex
\usepackage{tfrupee} % Include tfrupee package

\setlength{\headheight}{1cm} % Set the height of the header box
\setlength{\headsep}{0mm}  % Set the distance between the header box and the top of the text

\usepackage{gvv-book}
\usepackage{gvv}
\usepackage{cite}
\usepackage{amsmath,amssymb,amsfonts,amsthm}
\usepackage{algorithmic}
\usepackage{graphicx}
\usepackage{textcomp}
\usepackage{xcolor}
\usepackage{txfonts}
\usepackage{listings}
\usepackage{enumitem}
\usepackage{mathtools}
\usepackage{gensymb}
\usepackage{comment}
\usepackage[breaklinks=true]{hyperref}
\usepackage{tkz-euclide} 
\usepackage{listings}
% \usepackage{gvv}                                        
\def\inputGnumericTable{}                                 
\usepackage[latin1]{inputenc}                                
\usepackage{color}                                            
\usepackage{array}                                            
\usepackage{longtable}                                       
\usepackage{calc}                                             
\usepackage{multirow}                                         
\usepackage{hhline}                                           
\usepackage{ifthen}                                           
\usepackage{lscape}
\begin{document}

\bibliographystyle{IEEEtran}
\vspace{3cm}

\title{03/09/2020-Shift 1}
\author{EE24BTECH11021 - Eshan Ray}

% \maketitle
% \newpage
% \bigskip
{\let\newpage\relax\maketitle}

\renewcommand{\thefigure}{\theenumi}
\renewcommand{\thetable}{\theenumi}
\setlength{\intextsep}{10pt} % Space between text and floats

\begin{enumerate}
\setcounter{enumi}{15}
    \item The solution curve of the differential equation, $\brak{1+e^{-x}}\brak{1+y^2}\frac{dx}{dy}=y^2$, which passes through the point $\brak{0,1}$, is$
    colon$
        \begin{enumerate}
            \item $y^2=1+y\log_{e}\brak{\frac{1+e^{-x}}{2}}$
            \item $y^2+1=y\brak{\log_{e}\brak{\frac{1+e^{-x}}{2}}+2}$
            \item $y^2+1=y\brak{\log_{e}\brak{\frac{1+e^{x}}{2}}+2}$
            \item $y^2=1+y\log_{e}\brak{\frac{1+e^{x}}{2}}$
        \end{enumerate}
    \item The area in \brak{in sq.units} of the region $\cbrak{\brak{x,y}\colon 0\leq y\leq x^2+1,0\leq y\leq x+1,\frac{1}{2}\leq x\leq 2}$ is$\colon$
        \begin{enumerate}
            \item $\frac{23}{16}$
            \item $\frac{79}{16}$
            \item $\frac{23}{6}$
            \item $\frac{79}{24}$
        \end{enumerate}
    \item If $\alpha$ and $\beta$ are the roots of the equation $x^2+px+2=0$ and $\frac{1}{\alpha}$ and $\frac{1}{\beta}$ are the roots of the equation $2x^2+2qx+1=0$, then $\brak{\alpha-\frac{1}{\alpha}}\brak{\beta-\frac{1}{\beta}}\brak{\alpha + \frac{1}{\beta}}\brak{\beta +\frac{1}{\alpha}}$ is equal to$\colon$
        \begin{enumerate}
            \item $\frac{9}{4}\brak{9+p^2}$
            \item $\frac{9}{4}\brak{9+q^2}$
            \item $\frac{9}{4}\brak{9-p^2}$
            \item $\frac{9}{4}\brak{9-q^2}$
        \end{enumerate}
    \item The lines $\overrightarrow{r}=\brak{\hat{i}-\hat{j}}+l\brak{2\hat{i}+\hat{k}}$ and $\overrightarrow{r}=\brak{2\hat{i}-\hat{j}}+m\brak{\hat{i}+\hat{j}+\hat{k}}$
        \begin{enumerate}
            \item do not intersect for any values of $l$ and $m$
            \item intersect when $l=1$ and $m=2$
            \item intersect when $l=2$ and $m=\frac{1}{2}$
            \item intersect for all values of $l$ and $m$
        \end{enumerate}
    \item Let $\sbrak{t}$ denote the greatest integer $\leq t$. If for some $\lambda\in R-\cbrak{0,1}$,\\
    $\lim_{x \to 0} \abs{\frac{1-x+\abs{x}}{\lambda-x+\sbrak{x}}}=L$, then $L$ is equal to$\colon$
        \begin{enumerate}
            \item $0$
            \item $2$
            \item $\frac{1}{2}$
            \item $1$
        \end{enumerate}
    \item If $\lim_{x \to 0} \cbrak{\frac{1}{x^8}\brak{1-\cos{\frac{x^2}{2}}-\cos{\frac{x^2}{4}}+\cos{\frac{x^2}{2}}\cos{\frac{x^2}{4}}}}=2^{-k}$, then the value of $k$ is \dots 
    \item The diameter of the circle, whose centre lies on the line $x + y = 2$ in the first quadrant and which touches both the lines $x=3$ and $y=2$, is\dots
    \item The value of $\brak{0.16}^{\log_{2.5}\brak{\frac{1}{3}+\frac{1}{3^2}+\frac{1}{3^3}\dots to \infty}}$ is equal to \dots 
    \item Let $A=\myvec{x &1 \\1 &0},x\in R$ and $A^4=\sbrak{a_{ij}}$. If $a_{11}=109$, then $a_{22}$ is equal to \dots
    \item If $\brak{\frac{1+i}{1-i}}^{m/2}=\brak{\frac{1+i}{i-1}}^{n/3}=1,\brak{m,n\in N}$ then the greatest common divisor of the least values of $m$ and $n$ is \dots
    
\end{enumerate}
\end{document}
