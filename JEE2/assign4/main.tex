\let\negmedspace\undefined
\let\negthickspace\undefined
\documentclass[journal]{IEEEtran}
\usepackage[a5paper, margin=10mm, onecolumn]{geometry}
%\usepackage{lmodern} % Ensure lmodern is loaded for pdflatex
\usepackage{tfrupee} % Include tfrupee package

\setlength{\headheight}{1cm} % Set the height of the header box
\setlength{\headsep}{0mm}  % Set the distance between the header box and the top of the text

\usepackage{gvv-book}
\usepackage{gvv}
\usepackage{cite}
\usepackage{amsmath,amssymb,amsfonts,amsthm}
\usepackage{algorithmic}
\usepackage{graphicx}
\usepackage{textcomp}
\usepackage{xcolor}
\usepackage{txfonts}
\usepackage{listings}
\usepackage{enumitem}
\usepackage{mathtools}
\usepackage{gensymb}
\usepackage{comment}
\usepackage[breaklinks=true]{hyperref}
\usepackage{tkz-euclide} 
\usepackage{listings}
% \usepackage{gvv}                                        
\def\inputGnumericTable{}                                 
\usepackage[latin1]{inputenc}                                
\usepackage{color}                                            
\usepackage{array}                                            
\usepackage{longtable}                                       
\usepackage{calc}                                             
\usepackage{multirow}                                         
\usepackage{hhline}                                           
\usepackage{ifthen}                                           
\usepackage{lscape}
\begin{document}

\bibliographystyle{IEEEtran}
\vspace{3cm}

\title{26/02/2021-Shift 2}
\author{EE24BTECH11021 - Eshan Ray}

% \maketitle
% \newpage
% \bigskip
{\let\newpage\relax\maketitle}

\renewcommand{\thefigure}{\theenumi}
\renewcommand{\thetable}{\theenumi}
\setlength{\intextsep}{10pt} % Space between text and floats

\begin{enumerate}
    \item Let $L$ be a line obtained from the intersection of two planes $x+2y+z=6$ and $y+2z=4$. If point $P\brak{\alpha,\beta,\gamma}$ is the foot of perpendicular from $\brak{3,2,1}$ on $L$, then the value of $21\brak{\alpha,\beta,\gamma}$ equals $\colon$
        \begin{enumerate}
            \item $142$
            \item $68$
            \item $136$
            \item $102$
        \end{enumerate}
    \item The sum of the series $\sum_{n=1}^{\infty}\frac{n^2+6n+10}{\brak{2n+1}!}$ is equal to $\colon$
        \begin{enumerate}
            \item $\frac{41}{8}e+\frac{19}{8}e^{-1}-10$
            \item $-\frac{41}{8}e+\frac{19}{8}e^{-1}-10$
            \item $\frac{41}{8}e-\frac{19}{8}e^{-1}-10$
            \item $\frac{41}{8}e+\frac{19}{8}e^{-1}+10$
        \end{enumerate}
    \item Let $f\brak{x}$ be a differentiable function at $x=a$ with $f\prime\brak{a}=2$ and $f\brak{a}=4$. Then $\lim_{x \to a}\frac{xf\brak{a}-af\brak{x}}{x-a}$ equals$\colon$
        \begin{enumerate}
            \item $2a+4$
            \item $2a-4$
            \item $4-2a$
            \item $a+4$
        \end{enumerate}
    \item Let $A \brak{1, 4}$ and $B\brak{1, -5}$ be two points. Let $P$ be a point on the circle $\brak{x - 1}^2 + \brak{y - 1}^2 = 1$ such that $\brak{PA}^2+\brak{PB}^2$ have maximum value, then the points, $P, A$ and $B$ lie on $\colon$  
        \begin{enumerate}
            \item a parabola
            \item a straight line
            \item a hyperbola
            \item an ellipse
        \end{enumerate}
    \item If the locus of the mid-point of the line segment from the point $\brak{3, 2}$ to a point on the circle, $x^2 + y^2 = 1$ is a circle of the radius $r$, then $r$ is equal to : 
        \begin{enumerate}
            \item $\frac{1}{4}$
            \item $\frac{1}{2}$
            \item $1$
            \item $\frac{1}{3}$
        \end{enumerate}
    \item Let the slope of the tangent line to a curve at any point $P\brak{x,y}$ be given by $\frac{xy^2+y}{x}$. If the curve intersects the line $x+2y=4$ at $x=-2$, then the value of $y$, for which the point $\brak{3,y}$ lies on the curve, is$\colon$
        \begin{enumerate}
            \item $-\frac{18}{11}$
            \item $-\frac{18}{19}$
            \item $-\frac{4}{3}$
            \item $\frac{18}{35}$
        \end{enumerate}
    \item Let $A_1$ be the area bounded by the curves $y=\sin{x},y=\cos{x}$ and $y-axis$ in the first quadrant. Also, let $A_2$ be the area of the region bounded by the curves $y=\sin{x},y=\cos{x},x-axis$ and $x=\frac{\pi}{2}$ in the first quadrant.Then,
        \begin{enumerate}
            \item $A_1=A_2$ and $A_1+A_2=\sqrt{2}$
            \item $A_1\colon A_2=1\colon 2$ and $A_1+A_2=1$
            \item $2A_1=A_2$ and $A_1+A_2=1+\sqrt{2}$
            \item $A_1\colon A_2=1\colon \sqrt{2}$ and $A_1+A_2=1$
        \end{enumerate}
    \item If $0\textless a,b\textless 1$, and $\tan^{-1}{a}+\tan^{-1}{b}=\frac{\pi}{4}$, then the value of $\brak{a-b}-\brak{\frac{a^2+b^2}{2}}+\brak{\frac{a^3+b^3}{3}}-\brak{\frac{a^4+b^4}{4}}+\dots$ is$\colon$
        \begin{enumerate}
            \item $\log_{e}2$
            \item $\log_{e}\frac{e}{2}$
            \item $e$
            \item $e^2-1$
        \end{enumerate}
    \item Let $F_1\brak{A,B,C}=\brak{A\wedge \sim B}\vee \sbrak{\sim C\wedge\brak{A\vee B}}\vee \sim A$ and $F_2\brak{A,B}=\brak{A\vee B}\vee \brak{B\rightarrow \sim A}$ be two logical expressions. Then$\colon$
        \begin{enumerate}
            \item $F_1$ is not a tautology but $F_2$ is a tautology
            \item $F_1$ is a tautology but $F_2$ is not a tautology
            \item $F_1$ and $F_2$ both are tautologies
            \item Both $F_1$ and $F_2$ are not tautologies
        \end{enumerate}
    \item Consider the following system of equations $\colon$\\
    $x + 2y - 3z = a$\\
    $2x + 6y - 11 z = b$\\
    $x - 2y + 7z = c$,\\
    Where $a, b$ and $c$ are real constants. Then the system of equations $\colon$
        \begin{enumerate}
            \item has a unique solution when $5a = 2b + c$
            \item has infinite number of solutions when $5a = 2b + c$ 
            \item has no solution for all $a, b$ and $c$ 
            \item has a unique solution for all $a, b$ and $c$ 
        \end{enumerate}
    \item A seven digit number is formed using digit $3, 3, 4, 4, 4, 5, 5$. The probability, that number so formed is divisible by $2$, is $\colon$
        \begin{enumerate}
            \item $\frac{6}{7}$
            \item $\frac{4}{7}$
            \item $\frac{3}{7}$
            \item $\frac{1}{7}$
        \end{enumerate}
    \item If vectors $\overrightarrow{a_1}=x\hat{i}-\hat{j}+\hat{k}$ and $\overrightarrow{a_2}=\hat{i}+y\hat{j}+z\hat{k}$ are collinear, then a possible unit vector parallel to the vector parallel to the vector $x\hat{i}+y\hat{j}+z\hat{k}$ is $\colon$
        \begin{enumerate}
            \item $\frac{1}{\sqrt{2}}\brak{-\hat{j}+\hat{k}}$
            \item $\frac{1}{\sqrt{2}}\brak{\hat{i}-\hat{j}}$
            \item $\frac{1}{\sqrt{3}}\brak{\hat{i}-\hat{j}+\hat{k}}$
            \item $\frac{1}{\sqrt{3}}\brak{\hat{i}+\hat{j}-\hat{k}}$
        \end{enumerate}
    \item For $x\textgreater 0$, if $f\brak{x}=\int_{1}^{x}\frac{\log_{e}t}{1+t} \,dt$, then $f\brak{e}+f\brak{\frac{1}{e}}$ is equal to $\colon$
        \begin{enumerate}
            \item $\frac{1}{2}$
            \item $-1$
            \item $1$
            \item $0$
        \end{enumerate}
    \item Let $f\colon R\rightarrow R$ be defined as $f\brak{x}=
        \begin{cases}
            2\sin\brak{-\frac{\pi x}{2}} & \text{if } x\textless -1 \\
            \abs{ax^2+x+b}, & \text{if } -1\leq x\leq 1 \\
            \sin\brak{\pi x} & \text{if } x\textgreater 1
        \end{cases}
        $\\
        If $f\brak{x}$ is continuous on $R$, then $a+b$ equals $\colon$
        \begin{enumerate}
            \item $3$
            \item $-1$
            \item $-3$
            \item $1$
        \end{enumerate}
    \item Let $A=\cbrak{1,2,3\dots,10}$ and $f\colon A\rightarrow A$ be defined as $f\brak{k}=
        \begin{cases}
            k+1 & \text{if  k is odd}\\
            k & \text{if  k is even}
        \end{cases}    
    $
    Then the number of possible functions $g\colon A\rightarrow A$ such that $gof=f$ is $\colon$
        \begin{enumerate}
            \item $10^5$
            \item $\binom{10}{5}$
            \item $5^5$
            \item $5!$
        \end{enumerate}
\end{enumerate}
\end{document}
