\let\negmedspace\undefined
\let\negthickspace\undefined
\documentclass[journal]{IEEEtran}
\usepackage[a5paper, margin=10mm, onecolumn]{geometry}
%\usepackage{lmodern} % Ensure lmodern is loaded for pdflatex
\usepackage{tfrupee} % Include tfrupee package

\setlength{\headheight}{1cm} % Set the height of the header box
\setlength{\headsep}{0mm}  % Set the distance between the header box and the top of the text

\usepackage{gvv-book}
\usepackage{gvv}
\usepackage{cite}
\usepackage{amsmath,amssymb,amsfonts,amsthm}
\usepackage{algorithmic}
\usepackage{graphicx}
\usepackage{textcomp}
\usepackage{xcolor}
\usepackage{txfonts}
\usepackage{listings}
\usepackage{enumitem}
\usepackage{mathtools}
\usepackage{gensymb}
\usepackage{comment}
\usepackage[breaklinks=true]{hyperref}
\usepackage{tkz-euclide} 
\usepackage{listings}
% \usepackage{gvv}                                        
\def\inputGnumericTable{}                                 
\usepackage[latin1]{inputenc}                                
\usepackage{color}                                            
\usepackage{array}                                            
\usepackage{longtable}                                       
\usepackage{calc}                                             
\usepackage{multirow}                                         
\usepackage{hhline}                                           
\usepackage{ifthen}                                           
\usepackage{lscape}
\begin{document}

\bibliographystyle{IEEEtran}
\vspace{3cm}

\title{29/07/2022-Shift 2}
\author{EE24BTECH11021 - Eshan Ray}

% \maketitle
% \newpage
% \bigskip
{\let\newpage\relax\maketitle}

\renewcommand{\thefigure}{\theenumi}
\renewcommand{\thetable}{\theenumi}
\setlength{\intextsep}{10pt} % Space between text and floats

\begin{enumerate}
\setcounter{enumi}{15}
    \item Bag $I$ contains $3\, red,\,4\,black$ and $3\,white$ balls and Bag $II$ contains $2\,red,5\,black$ and $2\,white$ balls. One ball is transferred from Bag $I$ to Bag $II$ and then a ball is draw from Bag $II$. The ball so drawn is found to be $black$ in color. Then the probability, that the transferred ball is $red$, is $\colon$  
        \begin{enumerate}
            \item $\frac{4}{9}$
            \item $\frac{5}{18}$
            \item $\frac{1}{6}$
            \item $\frac{3}{10}$
        \end{enumerate}
    \item Let $S=\cbrak{z=x+ iy\colon\,\abs{z-1+i}\geq \abs{z},\abs{z}\textless 2,\abs{z+i}=\abs{z+1}}$. Then the set of all values of $x$, for which $w=2x+iy\in S$ for some $y\in R,$ is
        \begin{enumerate}
            \item $\left(-\sqrt{2},\frac{1}{2\sqrt{2}} \right]$
            \item $\left(-\frac{1}{\sqrt{2}},\frac{1}{4} \right]$
            \item $\left(-\sqrt{2},\frac{1}{2} \right]$
            \item $\left(-\frac{1}{\sqrt{2}},\frac{1}{2\sqrt{2}} \right]$
        \end{enumerate}
    \item Let $\overrightarrow{a,\overrightarrow{b},\overrightarrow{c}}$ be three coplanar concurrent vectors such that angles between two of them is same. If the product of their magnitudes is $14$ and $\brak{\overrightarrow{a}\times\overrightarrow{b}}\cdot\brak{\overrightarrow{b}\times\overrightarrow{c}}+\brak{\overrightarrow{b}\times\overrightarrow{c}}\cdot\brak{\overrightarrow{c}\times\overrightarrow{a}}+\brak{\overrightarrow{c}\times\overrightarrow{a}}\cdot\brak{\overrightarrow{a}\times\overrightarrow{b}}=168$ then $\abs{\overrightarrow{a}}+\abs{\overrightarrow{b}}+\abs{\overrightarrow{c}}$ is equal to $\colon$
        \begin{enumerate}
            \item $10$
            \item $14$
            \item $16$
            \item $18$
        \end{enumerate}
    \item The domain of the function $f\brak{x}=\sin^{-1}\brak{\frac{x^2-3x+2}{x^2+2x+7}}$ is $\colon$
        \begin{enumerate}
            \item $\left[1,\infty \right)$
            \item $\left(-1,2 \right]$
            \item $\left[-1,\infty \right)$
            \item $\left(-\infty,2 \right]$
        \end{enumerate}
    \item The statement $\brak{p\implies q}\vee \brak{p\implies r}$ is NOT equivalent to $\colon$
        \begin{enumerate}
            \item $\brak{p\wedge\brak{\sim r}}\implies q$
            \item $\brak{\sim q}\implies\brak{\brak{\sim r}\vee p}$
            \item $p\implies\brak{q\vee r}$
            \item $\brak{p\wedge\brak{\sim q}}\implies r$
        \end{enumerate}
    \item The sum and product of mean and variance of a binomial distribution are $82.5$ and $1350$ respectively. They the number of trials in the binomial distribution is $\colon$
    \item Let $\alpha,\beta\,\brak{\alpha\textgreater\beta}$ be the roots of the quadratic equation $x^2-x-4=0.$ If $P_n=\alpha^n-\beta^n,n\in N$, then $\frac{P_{15}P_{16}-P_{14}P_{16}-P^2_{15}+P_{14}P_{15}}{P_{13}P_{14}}$ is equal to \dots
    \item Let $x=\myvec{1\\1\\1}$ and $A=\myvec{-1&2&3\\0&1&6\\0&0&-1}$. For $k\in N$, if $x^\top A^k x=33$, then $k$ is equal to $\colon$
    \item The number of natural numbers lying between $1012$ and $23421$ that can be formed using the digits $2,3,4,5,6\brak{repetition\,of\,digits\,is\,not\,allowed}$ and divisible by $55$ is\dots
    \item If $\sum_{K=1}^{10}K^2\brak{\binom{10}{K}}^2=22000L$, then $L$ is equal to \dots
    \item If $\sbrak{t}$ denotes the greatest integer $\leq t$, then the number of points, at which the function $f\brak{x}=4\abs{2x+3}+8\sbrak{x+\frac{1}{2}}-12\sbrak{x+20}$ is not differentiable in the open interval $\brak{-20,20}$, is\dots
    \item If the tangent to the curve $x^3-x^2+x$ at the point $\brak{a,b}$ is also tangent to the curve $y=5x^2+2x-25$ at the point $\brak{2,-1}$ then $\abs{2a+9b}$ is equal to \dots
    \item Let $AB$ be a chord of length $12$ of the circle $$\brak{x-2}^2+\brak{y+1}^2=\frac{169}{4}$$
    If the tangents drawn to the circle at points $A$ and $B$ intersect at point $P$, then five times the distance of point $P$ from chord $AB$ is equal to \dots
    \item Let $\overrightarrow{a}$ and $\overrightarrow{b}$ be two vectors such that $\abs{\overrightarrow{a}+\overrightarrow{b}}^2=\abs{\overrightarrow{a}}^2+2\abs{\overrightarrow{b}}^2,\overrightarrow{a}\cdot\overrightarrow{b}=3$ and $\abs{\overrightarrow{a}\times\overrightarrow{b}}^2=75$. Then $\abs{\overrightarrow{a}}^2$ is equal to \dots
    \item Let $S=\cbrak{\brak{x,y}\in N\times N\colon9\brak{x-3}^2+16\brak{y-4}^2\leq 144}$\\
    and $T=\cbrak{\brak{x,y}\in R\times R\colon\brak{x-7}^2+\brak{y-4}^2\leq 36}$.\\
    The $n\brak{S\cap T}$ is equal to \dots
\end{enumerate}
\end{document}
