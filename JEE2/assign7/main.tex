\let\negmedspace\undefined
\let\negthickspace\undefined
\documentclass[journal]{IEEEtran}
\usepackage[a5paper, margin=10mm, onecolumn]{geometry}
%\usepackage{lmodern} % Ensure lmodern is loaded for pdflatex
\usepackage{tfrupee} % Include tfrupee package

\setlength{\headheight}{1cm} % Set the height of the header box
\setlength{\headsep}{0mm}  % Set the distance between the header box and the top of the text

\usepackage{gvv-book}
\usepackage{gvv}
\usepackage{cite}
\usepackage{amsmath,amssymb,amsfonts,amsthm}
\usepackage{algorithmic}
\usepackage{graphicx}
\usepackage{textcomp}
\usepackage{xcolor}
\usepackage{txfonts}
\usepackage{listings}
\usepackage{enumitem}
\usepackage{mathtools}
\usepackage{gensymb}
\usepackage{comment}
\usepackage[breaklinks=true]{hyperref}
\usepackage{tkz-euclide} 
\usepackage{listings}
% \usepackage{gvv}                                        
\def\inputGnumericTable{}                                 
\usepackage[latin1]{inputenc}                                
\usepackage{color}                                            
\usepackage{array}                                            
\usepackage{longtable}                                       
\usepackage{calc}                                             
\usepackage{multirow}                                         
\usepackage{hhline}                                           
\usepackage{ifthen}                                           
\usepackage{lscape}
\begin{document}

\bibliographystyle{IEEEtran}
\vspace{3cm}

\title{25/06/2022-Shift 1}
\author{EE24BTECH11021 - Eshan Ray}

% \maketitle
% \newpage
% \bigskip
{\let\newpage\relax\maketitle}

\renewcommand{\thefigure}{\theenumi}
\renewcommand{\thetable}{\theenumi}
\setlength{\intextsep}{10pt} % Space between text and floats

\begin{enumerate}
    \item Let a circle $C$ touch the lines $L_1\colon\,4x-3y+K_1=0$ and $L_2\colon\,4x-3y+K_2=0,K_1,K_2\in R.$ If a line passing through the centre of the circle $C$ intersects $L_1$ at $\brak{-1,2}$ and $L_2$ at $\brak{3,-6}$, then the equation of circle $C$ is 
        \begin{enumerate}
            \item $\brak{x-1}^2+\brak{y-2}^2=4$
            \item $\brak{x+1}^2+\brak{y-2}^2=4$
            \item $\brak{x-1}^2+\brak{y+2}^2=16$
            \item $\brak{x-1}^2+\brak{y-2}^2=16$
        \end{enumerate}
    \item The value of $\int_{0}^{\pi}\frac{e^{\cos{x}}\sin{x}}{\brak{1+\cos^2{x}}\brak{e^{\cos{x}}+e^{-\cos{x}}}} \, dx$ is equal to 
        \begin{enumerate}
            \item $\frac{\pi^2}{4}$
            \item $\frac{\pi^2}{2}$
            \item $\frac{\pi}{4}$
            \item $\frac{\pi}{2}$
        \end{enumerate}
    \item Let $a,b\,and\,c$ be the length of sides of triangle $ABC$ such that $\frac{a+b}{7}=\frac{b+c}{8}=\frac{c+a}{9}$. If $r$ and $R$ are the radius of incircle and radius of circumcircle of the triangle $ABC$, respectively then the value of $\frac{R}{r}$ is equal to
        \begin{enumerate}
            \item $\frac{5}{2}$
            \item $2$
            \item $\frac{3}{2}$
            \item $1$
        \end{enumerate}
    \item Let $f\colon\,N \to R$ be a function such that $f\brak{x+y}=2f\brak{x}f\brak{y}$ for natural numbers $x$ and $y$. If $f\brak{1}=2$, then the value of $\alpha$ for which
    $$\sum_{k=1}^{10}f\brak{\alpha+k}=\frac{512}{3}\brak{2^{20}-1}$$
    holds, is
        \begin{enumerate}
            \item $2$
            \item $3$
            \item $4$
            \item $6$
        \end{enumerate}
    \item Let $A$ be a $3\times 3$ real matrix such that $A\myvec{1\\1\\0}=\myvec{1\\1\\0};A\myvec{1\\0\\1}=\myvec{-1\\0\\1}$ and $A\myvec{0\\0\\1}=\myvec{1\\1\\2}$.
    If $X=\brak{x_1,x_2,x_3}^\top$ and $I$ is an identity matrix of order , then the system $\brak{A-2I}X=\myvec{4\\1\\1}$ has
        \begin{enumerate}
            \item no solution
            \item infinitely many solutions
            \item unique solution
            \item exactly two solutions
        \end{enumerate}
    \item Let $f\colon\, R\to R$ be defined as $f\brak{x}=x^3+x-5$. If $g\brak{x}$ is a function such that $f\brak{g\brak{x}}=x,\forall x\in R$, then $g\prime\brak{63}$ is equal to
        \begin{enumerate}
            \item $\frac{1}{49}$
            \item $\frac{3}{49}$
            \item $\frac{43}{49}$
            \item $\frac{91}{49}$
        \end{enumerate}
    \item Consider the following two propositions $\colon$\\
        $P1\colon\,\sim\brak{p \to \sim q}$\\
        $P2\colon\,\brak{p\wedge \sim q}\wedge\brak{\brak{\sim p}\vee q}$\\
        If the proposition $p \to \brak{\brak{\sim p}\vee q}$ is evaluated as FALSE, then $\colon$
        \begin{enumerate}
            \item $P1$ is TRUE and $P2$ is FALSE
            \item $P1$ is FALSE and $P2$ is TRUE
            \item Both $P1$ and $P2$ are FALSE 
            \item Both $P1$ and $P2$ are TRUE
        \end{enumerate}
    \item If $\frac{1}{2\cdot3^{10}}+\frac{1}{2^2\cdot3^{9}}+\dots +\frac{1}{2^{10}\cdot3}=\frac{K}{2^{10}\cdot3^{10}}$, then the remainder when $K$ is divided by $6$ is
        \begin{enumerate}
            \item $1$
            \item $2$
            \item $3$
            \item $5$
        \end{enumerate}
    \item Let $f\brak{x}$ be a polynomial function such that $f\brak{x}+f\prime\brak{x}+f\prime\prime\brak{x}=x^5+64$. Then, the value of $\lim_{x \to 1}\frac{f\brak{x}}{x-1}$
        \begin{enumerate}
            \item $-15$
            \item $-60$
            \item $60$
            \item $15$
        \end{enumerate}
    \item Let $E_1$ and $E_2$ be two events such that the conditional probabilities \\$P\brak{E_1\mid E_2}=\frac{1}{2},P\brak{E_2\mid E_1}=\frac{3}{4}$ and $P\brak{E_1\cap E_2}=\frac{1}{8} $  Then $\colon$
        \begin{enumerate}
            \item $P\brak{E_1\cap E_2}=P\brak{E_1}\cdot P\brak{E_2}$
            \item $ P\brak{E\prime_1\cap E\prime_2}=P\brak{E\prime_1}\cdot P\brak{E\prime_2} $
            \item $P\brak{E_1\cap E\prime_2}=P\brak{E_1}\cdot P\brak{E_2}$
            \item $P\brak{E\prime_1\cap E_2}=P\brak{E_1}\cdot P\brak{E_2}$
        \end{enumerate}
    \item Let $A=\myvec{0&-2\\2&0}$.If $M$ and $N$ are two matrices given by $M=\sum_{k=1}^{10}A^{2k}$ and $N=\sum_{k=1}^{10}A^{2k-1}$ then $MN^2$ is
        \begin{enumerate}
            \item a non-identity symmetric matrix
            \item a skew-symmetric matrix
            \item neither symmetric and skew-symmetric matrix
            \item an identity matrix
        \end{enumerate}
    \item Let $g\colon\,\brak{0,\infty}\to R$ be a differentiable function such that
        $$\int \brak{\frac{x\brak{\cos{x}-\sin{x}}}{e^x+1}+\frac{g\brak{x}\brak{e^x+1-xe^x}}{\brak{e^x+1}^2}} \, dx=\frac{xg\brak{x}}{e^x+1}+c,$$
        for all $x\textgreater 0$, where $c$ is an arbitrary constant. Then,
        \begin{enumerate}
            \item $g$ is decreasing in $\brak{0,\frac{\pi}{4}}$
            \item $g\prime$ is increasing in $\brak{0,\frac{\pi}{4}}$
            \item $g+g\prime$ is increasing in $\brak{o,\frac{\pi}{2}}$
            \item $g-g\prime$ is increasing in $\brak{0,\frac{\pi}{2}}$
        \end{enumerate}
    \item Let $f\colon\,R \to R$ and $g\colon\,R \to R$ be two functions defined by $f\brak{x}=\log_{e}\brak{x^2+1}-e^{-x}+1$ and $g\brak{x}=\frac{1-2e^{2x}}{e^x}$.Then, for which of the following range of $\alpha$, the inequality
        $$f\brak{g\brak{\frac{\brak{\alpha-1}^2}{3}}}\textgreater f\brak{g\brak{\alpha-\frac{5}{3}}}\, holds?$$ 
        \begin{enumerate}
            \item $\brak{2,3}$
            \item $\brak{-2,-1}$
            \item $\brak{1,2}$
            \item $\brak{-1,1}$
        \end{enumerate}
    \item Let $\overrightarrow{a}=a_1\hat{i}+a_2\hat{j}+a_3\hat{k}\,a_i\textgreater 0,i=1,2,3$ be a vector which makes equal angles with the coordinate axes $OX,OY$ and $OZ$. Also, let the projection of $\overrightarrow{a}$ on the vector $3\hat{i}+4\hat{j}$ be $7$. Let $\overrightarrow{b}$ be a vector obtained by rotating $\overrightarrow{a}$ with $90\degree$. If $\overrightarrow{a},\overrightarrow{b}$ and $x-axis$ are coplanar, then the projection of a vector $\overrightarrow{b}$ on $3\hat{i}+4\hat{j}$ is equal to
        \begin{enumerate}
            \item $\sqrt{7}$
            \item  $\sqrt{2}$
            \item $2$
            \item $7$
        \end{enumerate}
    \item Let $y=y\brak{x}$ be the solution of differential equation $\brak{x+1}y\prime-y=e^{3x}\brak{x+1}^2,$ with $y\brak{0}=\frac{1}{3}$. Then, the point $x=-\frac{4}{3}$ for the curve $y=y\brak{x}$ is $\colon$
        \begin{enumerate}
            \item not a critical point
            \item a point of local maxima
            \item a point of local minima
            \item a point of inflection
        \end{enumerate}        
\end{enumerate}
\end{document}
