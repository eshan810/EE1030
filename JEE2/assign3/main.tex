\let\negmedspace\undefined
\let\negthickspace\undefined
\documentclass[journal]{IEEEtran}
\usepackage[a5paper, margin=10mm, onecolumn]{geometry}
%\usepackage{lmodern} % Ensure lmodern is loaded for pdflatex
\usepackage{tfrupee} % Include tfrupee package

\setlength{\headheight}{1cm} % Set the height of the header box
\setlength{\headsep}{0mm}  % Set the distance between the header box and the top of the text

\usepackage{gvv-book}
\usepackage{gvv}
\usepackage{cite}
\usepackage{amsmath,amssymb,amsfonts,amsthm}
\usepackage{algorithmic}
\usepackage{graphicx}
\usepackage{textcomp}
\usepackage{xcolor}
\usepackage{txfonts}
\usepackage{listings}
\usepackage{enumitem}
\usepackage{mathtools}
\usepackage{gensymb}
\usepackage{comment}
\usepackage[breaklinks=true]{hyperref}
\usepackage{tkz-euclide} 
\usepackage{listings}
% \usepackage{gvv}                                        
\def\inputGnumericTable{}                                 
\usepackage[latin1]{inputenc}                                
\usepackage{color}                                            
\usepackage{array}                                            
\usepackage{longtable}                                       
\usepackage{calc}                                             
\usepackage{multirow}                                         
\usepackage{hhline}                                           
\usepackage{ifthen}                                           
\usepackage{lscape}
\begin{document}

\bibliographystyle{IEEEtran}
\vspace{3cm}

\title{04/09/2020-Shift 1}
\author{EE24BTECH11021 - Eshan Ray}

% \maketitle
% \newpage
% \bigskip
{\let\newpage\relax\maketitle}

\renewcommand{\thefigure}{\theenumi}
\renewcommand{\thetable}{\theenumi}
\setlength{\intextsep}{10pt} % Space between text and floats

\begin{enumerate}
\setcounter{enumi}{15}
    \item If $1+\brak{1-2^2\cdot1}+\brak{1-4^2\cdot3}+\brak{1-6^2\cdot5}+\dots +\brak{1-20^2\cdot 19}=\alpha-220\beta$, then the ordered pair $\brak{\alpha,\beta}$ is equal to $\colon$
        \begin{enumerate}
            \item $\brak{10,97}$
            \item $\brak{11,103}$
            \item $\brak{10,103}$
            \item $\brak{11,97}$
        \end{enumerate}
    \item Let $y=y\brak{x}$ be the solution of the differential equation, $xy\prime-y=x^2\brak{x\cos{x}+\sin{x}},x\textgreater 0$. If $y\brak{\pi}=\pi$, then $y\prime\prime\brak{\frac{\pi}{2}}+y\brak{\frac{\pi}{2}}$ is equal to $\colon$
        \begin{enumerate}
            \item $2+\frac{\pi}{2}$
            \item $1+\frac{\pi}{2}$
            \item $1+\frac{\pi}{2}+\frac{\pi^2}{4}$
            \item $2+\frac{\pi}{2}+\frac{\pi^2}{4}$
        \end{enumerate}
    \item The value of $\sum_{r=0}^{20}\binom{50-r}{6}$ is equal to $\colon$
        \begin{enumerate}
            \item $\binom{51}{7}+\binom{30}{7}$
            \item $\binom{51}{7}-\binom{30}{7}$
            \item $\binom{50}{7}-\binom{30}{7}$
            \item $\binom{50}{6}-\binom{30}{6}$
        \end{enumerate}
    \item TLet $f$ be a twice differential function on $\brak{1,6}$. If $f\brak{2}=8,f\prime\brak{2}=5,f\prime\brak{x}\geq 1$ and $f\prime\prime\brak{x}\geq 4$, for all $x\in \brak{1,6}$, then $\colon$
        \begin{enumerate}
            \item $f\brak{5}\leq 10$
            \item $f\prime\brak{5}+f\prime\prime\brak{5}\leq 20$
            \item $f\brak{5}+f\prime\brak{5}\geq 28$
            \item $f\brak{5}+f\prime\brak{5}\leq 26$
        \end{enumerate}
    \item If $\brak{a+\sqrt{2}b\cos{x}}\brak{a-\sqrt{2}b\cos{y}}=a^2-b^2$, where $a\textgreater b \textgreater 0$, then $\frac{dx}{dy}$ at $\brak{\frac{\pi}{4},\frac{\pi}{4}}$ is$\colon$
        \begin{enumerate}
            \item $\frac{a+b}{a-b}$
            \item $\frac{a-2b}{a+2b}$
            \item $\frac{a-b}{a+b}$
            \item $\frac{2a+b}{2a-b}$
        \end{enumerate}
    \item Suppose a differentiable function $f\brak{x}$ satisfies the identity $f\brak{x+y}=f\brak{x}+f\brak{y}+xy^2+x^2y$,for all real $x$ and $y$.If $\lim_{x \to 0}\frac{f\brak{x}}{x}=1$, then $f\prime \brak{3}$ is equal to\dots
    \item If the equation of a plane $P$, passing through the intersection of the planes, $x+4y-z+7=0$ and $3x+y+5z=8$ is $ax+by+6z=15$ for some $a, b\in R$, then the distance of the point $\brak{3,2,-1}$ from the plane $P$ is\dots units
    \item If the system of equations\\
    $x-2y+3z=9$\\
    $2x+y+z=b$\\
    $x-7y+az=24$, has infinitely many solutions, then $a-b$ is equal to \dots 
    \item Let $\brak{2x^2+3x+4}^{10}=\sum_{r=0}^{20}a_rx^r$. Then $\frac{a_7}{a_{13}}$ is equal to\dots
    \item The probability of a man hitting a target is $\frac{1}{10}$.The least number of shots required, so that the probability of his hitting the target at least once is greater than $\frac{1}{4}$, is\dots
    
\end{enumerate}
\end{document}
