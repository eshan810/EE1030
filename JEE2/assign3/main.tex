\let\negmedspace\undefined
\let\negthickspace\undefined
\documentclass[journal]{IEEEtran}
\usepackage[a5paper, margin=10mm, onecolumn]{geometry}
%\usepackage{lmodern} % Ensure lmodern is loaded for pdflatex
\usepackage{tfrupee} % Include tfrupee package

\setlength{\headheight}{1cm} % Set the height of the header box
\setlength{\headsep}{0mm}  % Set the distance between the header box and the top of the text

\usepackage{gvv-book}
\usepackage{gvv}
\usepackage{cite}
\usepackage{amsmath,amssymb,amsfonts,amsthm}
\usepackage{algorithmic}
\usepackage{graphicx}
\usepackage{textcomp}
\usepackage{xcolor}
\usepackage{txfonts}
\usepackage{listings}
\usepackage{enumitem}
\usepackage{mathtools}
\usepackage{gensymb}
\usepackage{comment}
\usepackage[breaklinks=true]{hyperref}
\usepackage{tkz-euclide} 
\usepackage{listings}
% \usepackage{gvv}                                        
\def\inputGnumericTable{}                                 
\usepackage[latin1]{inputenc}                                
\usepackage{color}                                            
\usepackage{array}                                            
\usepackage{longtable}                                       
\usepackage{calc}                                             
\usepackage{multirow}                                         
\usepackage{hhline}                                           
\usepackage{ifthen}                                           
\usepackage{lscape}
\begin{document}

\bibliographystyle{IEEEtran}
\vspace{3cm}

\title{04/09/2020-Shift 1}
\author{EE24BTECH11021 - Eshan Ray}

% \maketitle
% \newpage
% \bigskip
{\let\newpage\relax\maketitle}

\renewcommand{\thefigure}{\theenumi}
\renewcommand{\thetable}{\theenumi}
\setlength{\intextsep}{10pt} % Space between text and floats

\begin{enumerate}
\setcounter{enumi}{15}
    \item The mean and variance of $8$ observations are $10$ and $13.5$, respectively. If $6$ of these observations are $5, 7, 10, 12, 14, 15$, then the absolute difference of the remaining two observations is$\colon$
        \begin{enumerate}
            \item $3$
            \item $9$
            \item $7$
            \item $5$
        \end{enumerate}
    \item A survey shows that $63\%$ of the people in a city read newspaper $A$ whereas $76\%$ read newspaper $B$. If $x\%$ of the people read both the newspapers, then a possible value of $x$ can be$\colon$
        \begin{enumerate}
            \item $37$
            \item $29$
            \item $65$
            \item $55$
        \end{enumerate}
    \item Given the following two statements$\colon$\\
          $\brak{S_1}\colon\brak{q\vee p}\rightarrow \brak{p\leftrightarrow \sim q}$ is a tautology\\
          $\brak{S_2}\colon \sim q \wedge \brak{\sim p\leftrightarrow q}$ is a fallacy. Then$\colon$
        \begin{enumerate}
            \item only $\brak{S_1}$ is correct.
            \item both $\brak{S_1}$ and $\brak{S_2}$ are correct.
            \item only $\brak{S_2}$ is correct.
            \item both $\brak{S_1}$ and $\brak{S_2}$ are not correct.
        \end{enumerate}
    \item Two vertical poles $AB=15 m$ and $CD=10 m$ are standing apart on a horizontal ground with points $A$ and $C$ on the ground. If $P$ is the point of intersection of $BC$ and $AD$, then the height of $P \brak{in m}$ above the line $AC$ is:
        \begin{enumerate}
            \item $5$
            \item $\frac{20}{3}$
            \item $\frac{10}{3}$
            \item $6$
        \end{enumerate}
    \item If $\brak{a+\sqrt{2}b\cos{x}}\brak{a-\sqrt{2}b\cos{y}}=a^2-b^2$, where $a\textgreater b \textgreater 0$, then $\frac{dx}{dy}$ at $\brak{\frac{\pi}{4},\frac{\pi}{4}}$ is$\colon$
        \begin{enumerate}
            \item $\frac{a+b}{a-b}$
            \item $\frac{a-2b}{a+2b}$
            \item $\frac{a-b}{a+b}$
            \item $\frac{2a+b}{2a-b}$
        \end{enumerate}
    \item Suppose a differentiable function $f\brak{x}$ satisfies the identity $f\brak{x+y}=f\brak{x}+f\brak{y}+xy^2+x^2y$,for all real $x$ and $y$.If $\lim_{x \to 0}\frac{f\brak{x}}{x}=1$, then $f\prime \brak{3}$ is equal to\dots
    \item If the equation of a plane $P$, passing through the intersection of the planes, $x+4y-z+7=0$ and $3x+y+5z=8$ is $ax+by+6z=15$ for some $a, b\in R$, then the distance of the point $\brak{3,2,-1}$ from the plane $P$ is\dots units
    \item If the system of equations\\
    $x-2y+3z=9$\\
    $2x+y+z=b$\\
    $x-7y+az=24$, has infinitely many solutions, then $a-b$ is equal to \dots 
    \item Let $\brak{2x^2+3x+4}^{10}=\sum_{r=0}^{20}a_rx^r$. Then $\frac{a_7}{a_{13}}$ is equal to\dots
    \item The probability of a man hitting a target is $\frac{1}{10}$.The least number of shots required, so that the probability of his hitting the target at least once is greater than $\frac{1}{4}$, is\dots
    
\end{enumerate}
\end{document}
