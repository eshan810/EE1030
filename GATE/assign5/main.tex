\let\negmedspace\undefined
\let\negthickspace\undefined
\documentclass[journal]{IEEEtran}
\usepackage[a5paper, margin=10mm, onecolumn]{geometry}
%\usepackage{lmodern} % Ensure lmodern is loaded for pdflatex
\usepackage{tfrupee} % Include tfrupee package

\setlength{\headheight}{1cm} % Set the height of the header box
\setlength{\headsep}{0mm}  % Set the distance between the header box and the top of the text

\usepackage{gvv-book}
\usepackage{gvv}
\usepackage{cite}
\usepackage{amsmath,amssymb,amsfonts,amsthm}
\usepackage{algorithmic}
\usepackage{graphicx}
\usepackage{textcomp}
\usepackage{xcolor}
\usepackage{txfonts}
\usepackage{listings}
\usepackage{enumitem}
\usepackage{mathtools}
\usepackage{gensymb}
\usepackage{comment}
\usepackage[breaklinks=true]{hyperref}
\usepackage{tkz-euclide} 
\usepackage{listings}
% \usepackage{gvv}                                        
\def\inputGnumericTable{}                                 
\usepackage[latin1]{inputenc}                                
\usepackage{color}                                            
\usepackage{array}                                            
\usepackage{longtable}                                       
\usepackage{calc}                                             
\usepackage{multirow}                                         
\usepackage{hhline}                                           
\usepackage{ifthen}                                           
\usepackage{lscape}
\usepackage{tikz}
\usetikzlibrary{patterns}
\begin{document}

\bibliographystyle{IEEEtran}
\vspace{3cm}

\title{2010-EE- 53-65}
\author{EE24BTECH11021 - Eshan Ray}

% \maketitle
% \newpage
% \bigskip
{\let\newpage\relax\maketitle}

\renewcommand{\thefigure}{\theenumi}
\renewcommand{\thetable}{\theenumi}
\setlength{\intextsep}{10pt} % Space between text and floats

\begin{enumerate}
\setcounter{enumi}{52}
    \item Which of the following circuits is a realization of the above function $F$
\begin{enumerate}
    \item 
\item   \begin{circuitikz}
\tikzstyle{every node}=[font=\normalsize]
\draw (6.75,13.75) to[short] (7,13.75);
\draw (6.75,13.25) to[short] (7,13.25);
\draw (7,13.75) node[ieeestd nand port, anchor=in 1, scale=0.89](port){} (port.out) to[short] (8.75,13.5);
\draw (6.75,13.75) to[short] (6.75,13.25);
\draw (6.75,13.5) to[short] (5.5,13.5);
\draw (9.5,13.5) to[short] (9.75,13.5);
\draw (9.5,13) to[short] (9.75,13);
\draw (9.75,13.5) node[ieeestd and port, anchor=in 1, scale=0.89](port){} (port.out) to[short] (11.5,13.25);
\draw (12.5,12.75) to[short] (12.75,12.75);
\draw (12.5,12.25) to[short] (12.75,12.25);
\draw (12.75,12.75) node[ieeestd and port, anchor=in 1, scale=0.89](port){} (port.out) to[short] (14.5,12.5);
\draw (10.25,11.5) to[short] (10.5,11.5);
\draw (10.25,11) to[short] (10.5,11);
\draw (10.5,11.5) node[ieeestd and port, anchor=in 1, scale=0.89](port){} (port.out) to[short] (12.25,11.25);
\draw (11.5,13.25) to[short] (12.25,13.25);
\draw (12.25,13.25) to[short] (12.25,12.75);
\draw (12.25,12.75) to[short] (12.75,12.75);
\draw (12.25,11.25) to[short] (12.25,12.25);
\draw (12.25,12.25) to[short] (12.75,12.25);
\draw (8.75,13.5) to[short] (9.75,13.5);
\draw (6.5,13.5) to[short] (6.5,13);
\draw (9.5,13) to[short] (6.5,13);
\node [font=\normalsize] at (5.25,13.5) {X};
\node [font=\normalsize] at (10,11.5) {Y};
\node [font=\normalsize] at (10,11) {Z};
\node [font=\normalsize] at (14.75,12.5) {F};
\end{circuitikz}

\item             \begin{circuitikz}
\tikzstyle{every node}=[font=\normalsize]
\draw (12.5,14) to[short] (12.75,14);
\draw (12.5,13.5) to[short] (12.75,13.5);
\draw (12.75,14) node[ieeestd and port, anchor=in 1, scale=0.89](port){} (port.out) to[short] (14.5,13.75);
\draw (12.75,12.5) to[short] (13,12.5);
\draw (12.75,12) to[short] (13,12);
\draw (13,12.5) node[ieeestd and port, anchor=in 1, scale=0.89](port){} (port.out) to[short] (14.75,12.25);
\draw (16.5,13.25) to[short] (16.75,13.25);
\draw (16.5,12.75) to[short] (16.75,12.75);
\draw (16.75,13.25) node[ieeestd or port, anchor=in 1, scale=0.89](port){} (port.out) to[short] (18.5,13);
\draw (16.5,12.75) to[short] (15.25,12.75);
\draw (15.25,12.75) to[short] (15.25,12.25);
\draw (15.25,12.25) to[short] (14.75,12.25);
\draw (16.75,13.25) to[short] (15.25,13.25);
\draw (15.25,13.25) to[short] (15.25,13.75);
\draw (15.25,13.75) to[short] (14.5,13.75);
\draw (12.75,14) to[short] (11,14);
\draw (12.5,13.5) to[short] (11.75,13.5);
\draw (11.75,13.5) to[short] (11.75,14);
\node [font=\normalsize] at (10.75,14) {X};
\node [font=\normalsize] at (12.5,12.5) {Y};
\node [font=\normalsize] at (12.5,12) {Z};
\node [font=\normalsize] at (18.75,13) {F};
\end{circuitikz}
\item \begin{circuitikz}
\tikzstyle{every node}=[font=\normalsize]
\draw (10.75,14.5) to[short] (11,14.5);
\draw (10.75,14) to[short] (11,14);
\draw (11,14.5) node[ieeestd nand port, anchor=in 1, scale=0.89](port){} (port.out) to[short] (12.75,14.25);
\draw (13.75,14.25) to[short] (14,14.25);
\draw (13.75,13.75) to[short] (14,13.75);
\draw (14,14.25) node[ieeestd and port, anchor=in 1, scale=0.89](port){} (port.out) to[short] (15.75,14);
\draw (14,12.75) to[short] (14.25,12.75);
\draw (14,12.25) to[short] (14.25,12.25);
\draw (14.25,12.75) node[ieeestd nand port, anchor=in 1, scale=0.89](port){} (port.out) to[short] (16,12.5);
\draw (17,13.5) to[short] (17.25,13.5);
\draw (17,13) to[short] (17.25,13);
\draw (17.25,13.5) node[ieeestd and port, anchor=in 1, scale=0.89](port){} (port.out) to[short] (19,13.25);
\draw (13.75,14.25) to[short] (12.75,14.25);
\draw (13.75,13.75) to[short] (10.25,13.75);
\draw (10.75,14.5) to[short] (10.75,14);
\draw (10.75,14.25) to[short] (9.25,14.25);
\draw (10.25,13.75) to[short] (10.25,14.25);
\draw (16,12.5) to[short] (16.5,12.5);
\draw (17.25,13.5) to[short] (16.25,13.5);
\draw (16.25,13.5) to[short] (16.25,14);
\draw (16.25,14) to[short] (15.5,14);
\draw (16.5,12.5) to[short] (16.5,13);
\draw (16.5,13) to[short] (17,13);
\node [font=\normalsize] at (9,14.25) {X};
\node [font=\normalsize] at (13.75,12.75) {Y};
\node [font=\normalsize] at (13.75,12.25) {Z};
\node [font=\normalsize] at (19.25,13.25) {F};
\end{circuitikz}
\begin{circuitikz}
\tikzstyle{every node}=[font=\normalsize]
\draw (11.25,14) to[short] (11.5,14);
\draw (11.25,13.5) to[short] (11.5,13.5);
\draw (11.5,14) node[ieeestd nand port, anchor=in 1, scale=0.89](port){} (port.out) to[short] (13.25,13.75);
\draw (11.5,12) to[short] (11.75,12);
\draw (11.5,11.5) to[short] (11.75,11.5);
\draw (11.75,12) node[ieeestd nand port, anchor=in 1, scale=0.89](port){} (port.out) to[short] (13.5,11.75);
\draw (14.75,13.75) to[short] (15,13.75);
\draw (14.75,13.25) to[short] (15,13.25);
\draw (15,13.75) node[ieeestd and port, anchor=in 1, scale=0.89](port){} (port.out) to[short] (16.75,13.5);
\draw (15.25,11.75) to[short] (15.5,11.75);
\draw (15.25,11.25) to[short] (15.5,11.25);
\draw (15.5,11.75) node[ieeestd and port, anchor=in 1, scale=0.89](port){} (port.out) to[short] (17.25,11.5);
\draw (18.5,13) to[short] (18.75,13);
\draw (18.5,12.5) to[short] (18.75,12.5);
\draw (18.75,13) node[ieeestd or port, anchor=in 1, scale=0.89](port){} (port.out) to[short] (20.5,12.75);
\draw (15,13.75) to[short] (13.25,13.75);
\draw (14.75,13.25) to[short] (14,13.25);
\draw (13.5,11.75) to[short] (14,11.75);
\draw (14,11.75) to[short] (14,13.25);
\draw (15.5,11.75) to[short] (14.5,11.75);
\draw (14.5,11.75) to[short] (14.5,10.75);
\draw (14.5,10.75) to[short] (11.5,10.75);
\draw (11.5,10.75) to[short] (11.5,12);
\draw (11.25,14) to[short] (11.25,13.5);
\draw (11.25,13.75) to[short] (10.75,13.75);
\draw (11.5,11.75) to[short] (11,11.75);
\draw (16.75,13.5) to[short] (17.5,13.5);
\draw (17.5,13.5) to[short] (17.5,13);
\draw (17.5,13) to[short] (18.5,13);
\draw (17.25,11.5) to[short] (17.75,11.5);
\draw (17.75,11.5) to[short] (17.75,12.5);
\draw (17.75,12.5) to[short] (18.5,12.5);
\node [font=\normalsize] at (10.5,13.75) {X};
\node [font=\normalsize] at (10.75,11.75) {Y};
\node [font=\normalsize] at (15,11.25) {Z};
\node [font=\normalsize] at (20.75,12.75) {F};
\end{circuitikz}
    
    \end{enumerate}
        Statement for the linked Answer Questions $54$ and $55$\\
        The $L-C$ circuit shown in the figure has an inductance $L=1\,mH$ and a capacitance $C=10\,uF$

    \begin{circuitikz}
\tikzstyle{every node}=[font=\normalsize]
\draw (12.75,12.25) to[closing switch] (12.75,13.75);
\draw (12.75,13.75) to[L ] (14.75,13.75);
\draw [line width=0.6pt](15.5,13.75) to[C] (15.5,12);
\draw (15.5,13.75) to[short] (14.75,13.75);
\draw (12.75,12.25) to[short] (12.75,11.75);
\draw (12.75,11.75) to[short] (15.5,11.75);
\draw (15.5,11.75) to[short] (15.5,12.25);
\node [font=\normalsize] at (14.9,12.9) {C};
\node [font=\normalsize] at (16.5,12.9) {100 V};
\node [font=\normalsize] at (16.25,12.5) {+};
\node [font=\normalsize] at (16.25,13.25) {-};
\draw [->, >=Stealth] (14.25,13.5) -- (15,13.5);
\node [font=\normalsize] at (14.1,13.5) {i};
\node [font=\normalsize] at (13.75,14.25) {L};
\node [font=\normalsize] at (12,13) {t=0};
\end{circuitikz}

\item The initial current through the inductor is zero, while the initial capacitor voltage is $100\,V$. The switch is closed at $t=0$. The current $i$ through the circuit is $\colon$
    \begin{enumerate}
        \item $5\cos\brak{5\times10^4t}A$
        \item $5\sin\brak{10^4t}A$
        \item $10\cos\brak{5\times10^4t}A$
        \item $10\sin\brak{10^4t}A$
    \end{enumerate}
    \item The $L-C$ circuit of $Q54$ is used to communicate a thyristor, which is initially carrying a current of $5\,A$ as shown in the figure below. The values and initial conditions of $L$ and $C$ are same as in $Q54$. The switch is closed at $t=0$. If the forward drop is negligible, the time taken for the device to turn off is 

    \begin{circuitikz}
\tikzstyle{every node}=[font=\normalsize]
\draw (12.75,12.25) to[closing switch] (12.75,13.75);
\draw (12.75,13.75) to[L ] (14.75,13.75);
\draw [line width=0.6pt](15.5,13.75) to[C] (15.5,12);
\draw (15.5,13.75) to[short] (14.75,13.75);
\draw (12.75,12.25) to[short] (12.75,11.75);
\draw (15.5,11.75) to[short] (15.5,12.25);
\node [font=\normalsize] at (14.9,12.9) {C};
\node [font=\normalsize] at (16.5,12.9) {100 V};
\node [font=\normalsize] at (16.25,12.5) {+};
\node [font=\normalsize] at (16.25,13.25) {-};
\draw [->, >=Stealth] (14.25,13.5) -- (15,13.5);
\node [font=\normalsize] at (14.1,13.5) {i};
\node [font=\normalsize] at (13.75,14.25) {L};
\node [font=\normalsize] at (12,13) {t=0};
\draw (13.25,11.75) to[empty Zener diode] (15.25,11.75);
\draw (12.75,11.75) to[short] (13.25,11.75);
\draw (15.5,11.75) to[short] (15,11.75);
\draw (13,11.75) to[short] (11.75,11.75);
\draw (11.75,11.75) to[battery1] (11.75,10.5);
\draw [ line width=0.5pt](16.75,11.75) to[R] (16.75,10.25);
\draw (15.5,10.25) to[short] (11.75,10.25);
\draw (16.75,11.75) to[short] (15.5,11.75);
\draw (16.75,10.25) to[short] (15.25,10.25);
\draw (11.75,10.25) to[short] (11.75,10.75);
\draw [->, >=Stealth] (13.75,11.25) -- (14.75,11.25);
\node [font=\normalsize] at (14.25,11) {5 A};
\node [font=\normalsize] at (10.8,11.1) {100 V};
\node [font=\normalsize] at (17.5,11) {20  $\Omega$};
\end{circuitikz}
    \begin{enumerate}
        \item $52\,\mu s$
        \item $156\,\mu s$
        \item $312\,\mu s$
        \item $26\,\mu s$
    \end{enumerate}
    \item $25$ persons in a room. $15$ of them play hockey, $17$ of them play football and $10$ of them play both hockey and football. Then the number of persons playing neither football nor hockey is $\colon$
    \begin{enumerate}
        \item $2$
        \item $17$
        \item $13$
        \item $3$
    \end{enumerate} 
    \item $Choose\,the\,most\,appropriate\,word\,from\,the\,options\,given\,below\,to\,complete\,the\,\\following\,sentence\colon$ \\
    If we manage to $\dots\dots$ our natural resources, we would leave a better planet for our children.
    \begin{enumerate}
        \item uphold
        \item restrain
        \item cherish
        \item conserve
    \end{enumerate}
    \item $The\,below\,consists\,of\,a\,pair\,of\,related\,words\,followed\,by\,four\,pairs\,of\,words.\,Select\,the\\\,pair\,that\,best\,expresses\,the\,relation\,in\,the\,original\,pair.$\\
    Unemployed $\colon$ Worker
    \begin{enumerate}
        \item fallow $\colon$ land
        \item unaware $\colon$ sleeper
        \item wit $\colon$ jester
        \item renovated $\colon$ house
    \end{enumerate}
    \item $Which\,of\,the\,following\,options\,is\,the\,closest\,in\,meaning\,to\,the\,word\,below\,\colon$\\
    Circuitous
    \begin{enumerate}
        \item cyclic
        \item indirect
        \item confusing
        \item crooked
    \end{enumerate}
    \item $Choose\,the\,most\,appropriate\,word\,from\,the\,options\,given\,below\,to\,complete\,the\,following\\\,sentance$\\
    His rather casual remarks on politics $\dots\dots$ his lack of seriousness about the subject.
    \begin{enumerate}
        \item masked
        \item belied
        \item betrayed
        \item suppressed
    \end{enumerate}
    \item Hari $\brak{H}$, Gita $\brak{G}$, Irfan $\brak{I}$ and Saira $\brak{S}$ are siblings $\brak{i\cdot e\cdot\, brothers\, and\, sisters}$. All were born on $1^{st}$ January. The age difference between any two successive siblings $\brak{that\, is\, born\, one\, after\, another}$ is less than $3$ years. Given the following facts $\colon$
    \begin{enumerate}
     \item[i.] $Hari's\, age + Gita's\, age\, >\, Irfan's\, age\, + Saira's\, age$. 
    \item[ii.] The age difference between Gita and Saira is $1$ year. However, Gita is not the oldest and Saira is not the youngest.
    \item[iii.] There are no twins. 
    \end{enumerate}
    In what order were they born $\brak{oldest\, first}$?
    \begin{enumerate}
        \item $HSIG$
        \item $SGHI$
        \item $IGSH$
        \item $IHSG$
    \end{enumerate}
    \item $5$ skilled workers can build a wall in $20$ days; $8$ semi-skilled workers can build a wall in $25$ days; $10$ unskilled workers can build a wall in $30$ days. If a team has $2$ skilled, $6$ semi-skilled and $5$ unskilled workers, how long will it take to build the wall?
    \begin{enumerate}
        \item $20$ days
        \item $18$ days
        \item $16$ days
        \item $15$ days
    \end{enumerate}
    \item Modern warfare has changed from large scale clashes of armies to suppression of civilian populations. Chemical agents that do their work silently appear to be suited to such warfare; and regretfully, there exist people in military establishments who think that chemical agents are useful tools for their cause.\\ \\
    $Which\,of\,the\,following\,statemens\,best\,sums\,up\,the meaning\,of\,above\,passage\colon$
    \begin{enumerate}
        \item Modern warfare has resulted in civil strife
        \item Chemical agents are useful in modern warfare
        \item Use of chemical agents in warfare would be undesirable.
        \item People in military establishments like to use chemical agents in war.
    \end{enumerate}
    \item Given digits $2,2,3,3,3,4,4,4,4$ how many distinct $4$ digit numbers greater than $3000$ can be formed?
    \begin{enumerate}
        \item $50$
        \item $51$
        \item $52$
        \item $53$
    \end{enumerate}
    \item If $137+276=435$ how many is $731+672$?
    \begin{enumerate}
        \item $534$
        \item $1403$
        \item $1623$
        \item $1513$
    \end{enumerate}
\end{enumerate}
\end{document}
