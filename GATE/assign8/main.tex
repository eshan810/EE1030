\let\negmedspace\undefined
\let\negthickspace\undefined
\documentclass[journal]{IEEEtran}
\usepackage[a5paper, margin=10mm, onecolumn]{geometry}
%\usepackage{lmodern} % Ensure lmodern is loaded for pdflatex
\usepackage{tfrupee} % Include tfrupee package

\setlength{\headheight}{1cm} % Set the height of the header box
\setlength{\headsep}{0mm}  % Set the distance between the header box and the top of the text

\usepackage{gvv-book}
\usepackage{gvv}
\usepackage{cite}
\usepackage{amsmath,amssymb,amsfonts,amsthm}
\usepackage{algorithmic}
\usepackage{graphicx}
\usepackage{textcomp}
\usepackage{xcolor}
\usepackage{txfonts}
\usepackage{listings}
\usepackage{enumitem}
\usepackage{mathtools}
\usepackage{gensymb}
\usepackage{comment}
\usepackage[breaklinks=true]{hyperref}
\usepackage{tkz-euclide} 
\usepackage{listings}
% \usepackage{gvv}                                        
\def\inputGnumericTable{}                                 
\usepackage[latin1]{inputenc}                                
\usepackage{color}                                            
\usepackage{array}                                            
\usepackage{longtable}                                       
\usepackage{calc}                                             
\usepackage{multirow}                                         
\usepackage{hhline}                                           
\usepackage{ifthen}                                           
\usepackage{lscape}
\usepackage{tikz}
\begin{document}

\bibliographystyle{IEEEtran}
\vspace{3cm}

\title{2013-MA- 27-39}
\author{EE24BTECH11021 - Eshan Ray}

% \maketitle
% \newpage
% \bigskip
{\let\newpage\relax\maketitle}

\renewcommand{\thefigure}{\theenumi}
\renewcommand{\thetable}{\theenumi}
\setlength{\intextsep}{10pt} % Space between text and floats

\begin{enumerate}
\setcounter{enumi}{26}
    \item Let $f,g\colon\sbrak{0,1}\to R$ be defined by
        $$f\brak{x}=
        \begin{cases}
            x \text{ if} x=\frac{1}{n}\,for\,n\in N\\
            0 \text{ otherwise}
        \end{cases}
        $$ and
        $$g\brak{x}
        \begin{cases}
            1 \text{ if } x\in Q\cap\sbrak{0,1}\\
            0 \text{ otherwise}
        \end{cases}
        $$ Then
    \begin{enumerate}
        \item Both $f$ and $g$ are Riemann integral
        \item $f$ is a Riemann integrable and $g$ is Lebesgue integrable
        \item $g$ is Riemann integrable and $f$ is Lebesgue integrable
        \item Neither $f$ nor $g$ is Riemann integrable
    \end{enumerate}
    \item Consider the following linear programming problem$\colon$\\ \\
  $  \begin{array}{cr}
\text{Maximize} & x + 3y + 6z - w \\
\text{subject to} & 5x + y + 6z + 7w \leq 20, \\
& 6x + 2y + 2z + 9w \leq 40, \\
& x \geq 0, \; y \geq 0, \; z \geq 0, \; w \geq 0.
\end{array} $ \\
Then the optimal value is \dots
    \item Suppose $X$ is real-valued real random variable. Which of the following values $CANNOT$ be attained by $E\sbrak{X}$ and $E\sbrak{X^2},$ respectively?
    \begin{enumerate}
        \item $0$ and $1$ 
        \item $2$ and $3$ 
        \item $\frac{1}{2}$ and $\frac{1}{3}$
        \item $2$ and $5$
    \end{enumerate}
    \item Which of the following subsets of $R^2$ is NOT compact?
    \begin{enumerate}
        \item $\cbrak{\brak{x,y}\in R^2\colon\,-1\leq x\leq1,\,y=\sin x}$
        \item $\cbrak{\brak{x,y}\in R^2\colon\,-1\leq y\leq1,\,y=x^8-x^3-1}$
        \item $\cbrak{\brak{x,y}\in R^2\colon\,y=0,\,\sin\brak{e}^{-x}=0}$
        \item $\cbrak{\brak{x,y}\in R^2\colon\, x\textgreater 0,y=\sin\brak{\frac{1}{x}}}\cap\cbrak{\brak{x,y}\in R^2\colon x\textgreater0,y=\frac{1}{x}}$
    \end{enumerate}
    \item Let $M$ be the real space vector of $2\times3$ matrices with real entries. Let $t\colon M\to M$ be defined by
    $$T\brak{\myvec{x_1&x_2&x_3\\x_4&x_5&x_6}}=\myvec{-x_6&x_4&x_1\\x_3&x_5&x_2}$$
    The determinant of $T$ is \dots
    \item Let $\mathcal{H}$ be a Hilbert space and let $\cbrak{e_n\colon n\geq1}$ be an orthonormal basis of $\mathcal{H}$. Suppose $T\colon\mathcal{H}\to \mathcal{H}$ is a bounded linear operator. Which of the following $CANNOT$ be true?
     \begin{enumerate}
        \item $T\brak{e_n}= e_1$ for all $n\geq1$
        \item $T\brak{e_n}= e_{n+1}$ for all $n\geq1$
        \item $T\brak{e_n}= \frac{n+1}{n}e_n$ for all $n\geq1$
        \item $T\brak{e_n}= e_{n-1}$ for all $n\geq2$ and $T\brak{e_1}=0$
    \end{enumerate}
    \item The value of the limit 
            $$\lim_{n\to\infty}\frac{2^{-n^2}}{\sum_{k=n+1}^{\infty}2^{-k^2}}$$
            is
        \begin{enumerate}
            \item $0$
            \item some $c\in \brak{0,1}$
            \item $1$
            \it'$\infty$
        \end{enumerate}
    \item Let $f\colon C\backslash \cbrak{3i}\to C$ be defined by $f\brak{z}=\frac{z-i}{iz+3}$. Which of the following statements about $f$ is FALSE ?
    \begin{enumerate}
        \item $f$ is conformal on $C\backslash\cbrak{3i}$
        \item $f$ maps circles in $C\backslash\cbrak{3i}$ onto circles in $C$
        \item All the fixed points of $f$ are in the region $\cbrak{z\in C\colon Im\brak{z}\textgreater 0}$
        \item There is no straight line in $C\backslash\cbrak{3i}$ which is mapped onto a straight line in $C$ by $f$ 
    \end{enumerate}
    \item The matrix $A=\myvec{1&2&0\\1&3&1\\0&1&3}$ can be decomposed uniquely into the product $A=\,LU,$ where $L=\myvec{1&0&0\\l_{21}&1&0\\l_{31}&l_{32}&1}$ and $U=\myvec{u_{11}&u_{12}&u_{13}\\0&u_{22}&u_{23}\\0&0&u_{33}}$. The solution of the system $LX=\myvec{1&2&2}^t$ is 
    \begin{enumerate}
        \item $\myvec{1&1&1}^t$
        \item $\myvec{1&1&0}^t$
        \item $\myvec{0&1&1}^t$
        \item $\myvec{1&0&1}^t$
    \end{enumerate}
    \item Let $S=\cbrak{x\in R\colon x\geq0,\sum_{n=1}^{\infty}x^{\sqrt{n}}\textless \infty}$. Then the supremum of $S$ is
    \begin{enumerate}
        \item $1$
        \item $\frac{1}{e}$
        \item $0$
        \item $\infty$
    \end{enumerate}
    \item The image of the region $\cbrak{z\in C\colon Re\brak{z}\textgreater Im\brak{z}\textgreater0}$ under the mapping $z \mapsto e^{z^2}$ is
        \begin{enumerate}
            \item $\cbrak{w\in C\colon Re\brak{w}\textgreater0,Im\brak{w}\textgreater0}$
            \item $\cbrak{w\in C\colon Re\brak{w}\textgreater0,Im\brak{w}\textgreater0,\abs{w}\textgreater1}$
            \item $\cbrak{w\in C\colon \abs{w}\textgreater1}$
            \item $\cbrak{w\in C\colon Im\brak{w}\textgreater0,\abs{w}\textgreater1}$
        \end{enumerate}
    \item Which of the following groups contain a unique normal subgroup of order four?
            \begin{enumerate}
                \item $Z_2\oplus Z_4$
                \item The dihedral group, $D_4$, of order eight
                \item The quaternion group, $Q_8$
                \item $Z_2\oplus Z_2\oplus Z_2$
            \end{enumerate}
    \item Let $B$ be a real symmetric positive-definite  $n\times n$ matrix. Consider the inner product on $R^n$ defined by $\langle X,Y\rangle=Y^t BX$. Let $A$ be an $n\times n$ real matrix and let $T\colon R^n\to R^n$ be the linear operator defined by $T\brak{X}=AX$ for all $X\in R^n$. If $S$ is the adjoint of $T$, then $S\brak{X}=CX$ for all $X\in R^n,$ where $C$ is the matrix
    \begin{enumerate}
        \item $B^{-1}A^t B$
        \item $BA^tB^{-1}$
        \item $B^{-1}AB$
        \item $A^t$
    \end{enumerate}
\end{enumerate}
\end{document}
