\let\negmedspace\undefined
\let\negthickspace\undefined
\documentclass[journal]{IEEEtran}
\usepackage[a5paper, margin=10mm, onecolumn]{geometry}
%\usepackage{lmodern} % Ensure lmodern is loaded for pdflatex
\usepackage{tfrupee} % Include tfrupee package

\setlength{\headheight}{1cm} % Set the height of the header box
\setlength{\headsep}{0mm}  % Set the distance between the header box and the top of the text

\usepackage{gvv-book}
\usepackage{gvv}
\usepackage{cite}
\usepackage{amsmath,amssymb,amsfonts,amsthm}
\usepackage{algorithmic}
\usepackage{graphicx}
\usepackage{textcomp}
\usepackage{xcolor}
\usepackage{txfonts}
\usepackage{listings}
\usepackage{enumitem}
\usepackage{mathtools}
\usepackage{gensymb}
\usepackage{comment}
\usepackage[breaklinks=true]{hyperref}
\usepackage{tkz-euclide} 
\usepackage{listings}
% \usepackage{gvv}                                        
\def\inputGnumericTable{}                                 
\usepackage[latin1]{inputenc}                                
\usepackage{color}                                            
\usepackage{array}                                            
\usepackage{longtable}                                       
\usepackage{calc}                                             
\usepackage{multirow}                                         
\usepackage{hhline}                                           
\usepackage{ifthen}                                           
\usepackage{lscape}
\usepackage{tikz}
\begin{document}

\bibliographystyle{IEEEtran}
\vspace{3cm}

\title{2016-PH- 14-26}
\author{EE24BTECH11021 - Eshan Ray}

% \maketitle
% \newpage
% \bigskip
{\let\newpage\relax\maketitle}

\renewcommand{\thefigure}{\theenumi}
\renewcommand{\thetable}{\theenumi}
\setlength{\intextsep}{10pt} % Space between text and floats

\begin{enumerate}
\setcounter{enumi}{13}
    \item There are four electrons in $3d$ shell of an isolated atom. The total magnetic moment of the atom in units in Bohr magneton is \dots
    \item Which of the following transitions is NOT allowed in the case of an atom, according to the electric dipole radiation selection rule?
    \begin{enumerate}
        \item $2s-1s$
        \item $2p-1s$
        \item $2p-2s$
        \item $3d-2p$
    \end{enumerate}
    \item In the $SU\brak{3}$ quark model, the triplet of mesons $\brak{\pi^+,\pi^0,\pi^-}$ has
    \begin{enumerate}
        \item Isospin=$0$, Strangeness =$0$
        \item Isospin=$1$, Strangeness =$0$
        \item Isospin=$\frac{1}{2}$, Strangeness =$+1$
        \item Isospin=$\frac{1}{2}$, Strangeness =$-1$
    \end{enumerate}
    \item The magnitude of the magnetic dipole moment associated with a square shaped loop carrying a steady current $I$ is $m$. If this loop is changed to a circular shape with the same current $I$ passing through it, the magnetic dipole moment becomes $\frac{pm}{\pi}$. The value of $p$ is \dots
    \item The total power emitted by a spherical black body of Radius $R$ at a temperature $T$ is $P_1$. Let, $P_2$ be the total power emitted by another spherical black body  of radius $\frac{R}{2}$ kept at an temperature $2T$. The ratio, $\frac{P_1}{P_2}$ is \dots. \brak{Give\,your\,answer\,upto\,two\,decimal\,places}
    \item The entropy $S$ of a system of $N$ spins, which may align either in the upward or in the downward direction, is given by $S=-k_BN\sbrak{p\ln{p}+\brak{1-p}\ln\brak{1-p}}$. Here, $k_B$ is the Boltzmann constant. The probability of alignment in the upward direction is $p$ .The value of $p$, at which the entropy is maximum, is \dots. \brak{Give\,your\,answer\,upto\,one\,decimal\,place}
    \item For a system at constant temperature and volume, which of the following statements is correct at equilibrium?
    \begin{enumerate}
        \item The Helmholtz free energy attains a local minimum.
        \item The Helmholtz free energy attains a local maximum.
        \item The Gibbs free energy attains a local minimum.
        \item The Gibbs free energy attains a local maximum.
    \end{enumerate}
    \item $N$ atoms of an ideal gas are enclosed in a container of volume $V$. The volume of the container is changed to $4V$, while keeping the total energy constant. The change in the entropy of the gas, in units of $Nk_n\ln 2,$ is \dots, where $k_B$ is the Boltzmann constant.
    \item Which of the following is an analytic function of $z$ everywhere in the complex plane?
    \begin{enumerate}
        \item $z^2$
        \item $\brak{z^{\cdot}}^2$
        \item $\abs{z}^2$
        \item $\sqrt{z}$
    \end{enumerate}
    \item In a Young's double slit experiment using light, the apparatus has two slits of unequal widths. When only $slit-1$ is open, the maximum observed intensity on the screen is $4I_0$. When only $slit-2$ is open, the maximum observed intensity is $I_0$. When both the slits are open, an interference pattern appears on the screen. The ratio of the intensity of principal maximum to that of the nearest minimum is \dots
    \item Consider a metal which obeys the Sommerfeld model exactly. If $E_F$ is the Fermi energy of the metal at $T=0\,K$ and $R_H$ is the Hall coefficient , which of the following statements is correct ?
    \begin{enumerate}
        \item $R_H\propto E_F^{\frac{3}{2}}$
        \item $R_H\propto E_F^{\frac{2}{3}}$
        \item $R_H\propto E_F^{-\frac{3}{2}}$
        \item $R_H$ is independent of $E_F$
    \end{enumerate}
    \item A one-dimensional linear chain of atoms contains two types of atoms of masses $m_1$ and $m_2\brak{where\,m_2\textgreater m_1}$, arranged alternately. The distance between successive atoms is the same. Assume that the harmonic approximation is valid. At the first Brillouin zone boundary , which of the following statements is correct?
    \begin{enumerate}
        \item The atoms of mass $m_2$ are at rest in optical mode, while they vibrate in acoustical mode.
        \item The atoms of mass $m_1$ are at rest in optical mode, while they vibrate in acoustical mode.
        \item Both types of atoms vibrate with equal amplitudes in the optical as well as in the acoustical modes.
        \item Both types of atoms vibrate, but with unequal, non-zero amplitudes in the optical as well as in the acoustical modes.
    \end{enumerate}
    \item Which of the following operators is Hermitian?
    \begin{enumerate}
        \item $\frac{d}{dx}$
        \item $\frac{d^2}{dx^2}$
        \item $i\frac{d^2}{dx^2}$
        \item $\frac{d^3}{dx^3}$
    \end{enumerate}
\end{enumerate}
\end{document}
