\let\negmedspace\undefined
\let\negthickspace\undefined
\documentclass[journal]{IEEEtran}
\usepackage[a5paper, margin=10mm, onecolumn]{geometry}
%\usepackage{lmodern} % Ensure lmodern is loaded for pdflatex
\usepackage{tfrupee} % Include tfrupee package

\setlength{\headheight}{1cm} % Set the height of the header box
\setlength{\headsep}{0mm}  % Set the distance between the header box and the top of the text

\usepackage{gvv-book}
\usepackage{gvv}
\usepackage{cite}
\usepackage{amsmath,amssymb,amsfonts,amsthm}
\usepackage{algorithmic}
\usepackage{graphicx}
\usepackage{textcomp}
\usepackage{xcolor}
\usepackage{txfonts}
\usepackage{listings}
\usepackage{enumitem}
\usepackage{mathtools}
\usepackage{gensymb}
\usepackage{comment}
\usepackage[breaklinks=true]{hyperref}
\usepackage{tkz-euclide} 
\usepackage{listings}
% \usepackage{gvv}                                        
\def\inputGnumericTable{}                                 
\usepackage[latin1]{inputenc}                                
\usepackage{color}                                            
\usepackage{array}                                            
\usepackage{longtable}                                       
\usepackage{calc}                                             
\usepackage{multirow}                                         
\usepackage{hhline}                                           
\usepackage{ifthen}                                           
\usepackage{lscape}
\begin{document}

\bibliographystyle{IEEEtran}
\vspace{3cm}

\title{2007-MA- 1-17}
\author{EE24BTECH11021 - Eshan Ray}

% \maketitle
% \newpage
% \bigskip
{\let\newpage\relax\maketitle}

\renewcommand{\thefigure}{\theenumi}
\renewcommand{\thetable}{\theenumi}
\setlength{\intextsep}{10pt} % Space between text and floats

\begin{enumerate}
    \item Consider $R^2$ with the usual topology. Let $S=\cbrak{\brak{x,y}\in R^2\colon x\, is\, an\, integer}$. Then $S $ is 
    \begin{enumerate}
        \item open but NOT closed
        \item both open and closed
        \item neither open nor closed
        \item closed but not open
    \end{enumerate}
    \item Suppose $X=\cbrak{\alpha,\beta,\gamma}$. Let
    $$\mathcal{J}_1=\cbrak{\phi,X,\cbrak{\alpha},\cbrak{\alpha,\beta}}\,and\, \mathcal{J}_2= \cbrak{\phi,X,\cbrak{\alpha},\cbrak{\beta,\gamma}}.$$
    Then
    \begin{enumerate}
        \item both $\mathcal{J}_1\cap\mathcal{J}_2$ and $\mathcal{J}_1\cup \mathcal{J}_2$ are topologies
        \item neither $\mathcal{J}_1\cap\mathcal{J}_2$ nor $\mathcal{J}_1\cup \mathcal{J}_2$ is a  topologies
        \item $\mathcal{J}_1\cup \mathcal{J}_2$ is a topology but $\mathcal{J}_1\cap\mathcal{J}_2$ is NOT a topology
        \item  $\mathcal{J}_1\cap\mathcal{J}_2$ is a topology but $\mathcal{J}_1\cup \mathcal{J}_2$ is NOT a  topology
    \end{enumerate}
    \item For a positive integer $n$, let $f_n\colon R\to R$ be defined by \\
    $$f_n\brak{x}=
    \begin{cases}
    \frac{1}{4n+5} \text{, if } 0\leq x\leq n,\\
    0 \text{, otherwise}
    \end{cases}
    $$
    Then $\cbrak{f_n\brak{x}}$ converges to zero 
        \begin{enumerate}
            \item uniformly but NOT in $L^1$ norm
            \item uniformly and also in $L^1$ norm
            \item pointwise but NOT uniformly
            \item in $L^1$ norm but NOT pointwise
        \end{enumerate}
    \item Let $P_1$ and $P_2$ be two projection operations on a vector space. Then
        \begin{enumerate}
            \item $P_1+P_2$ is a projection if $P_1P_2=P_2P_1=0$
            \item $P_1-P_2$ is a projection if $P_1P_2=P_2P_1=0$
            \item $P_1+P_2$ is a projection
            \item $P_1=P_2$ is a projection
        \end{enumerate}
    \item Consider the system of linear equations\\
            $x+y+z=3$\\
            $x-y-z=4$\\
            $x-5y+kz=6$.\\
            Then the value of $k$ for which this  system has an infinite number of solutions is
            \begin{enumerate}
                \item $k=-5$
                \item $k=0$
                \item $k=1$
                \item $k=3$
            \end{enumerate}
    \item Let
        $$A=\myvec{1&1&1\\2&2&3\\x&y&z}$$
        and let $V=\cbrak{\brak{x,y,z}\in R^3\colon det\brak{A}=0}$. Then the dimension of $V$ equals
        \begin{enumerate}
            \item infinite
            \item $0$
            \item $1$
            \item $2$
        \end{enumerate}
    \item Let $S=\cbrak{0}\cup\cbrak{\frac{1}{4n+7}\colon n=1,2,\dots}$. Then the number of analytic functions which vanish only on $S$ is 
    \begin{enumerate}
        \item $0$
        \item $1$
        \item $2$
        \item $3$
    \end{enumerate}
    \item It is given that $\sum_{n=0}^{\infty} a_nz^n$ converges at $z=3+i4$. Then the radius of convergence of the power series $\sum_{n=0}^{\infty} a_nz^n$ is
    \begin{enumerate}
        \item $\leq 5$
        \item $\geq 5$
        \item $\textless 5$
        \item $\textgreater 5$
    \end{enumerate}
    \item The value of $\alpha$ for which $G=\cbrak{\alpha,1,3,9,19,27}$ is a cyclic group under multiplication modulo $56$ is 
    \begin{enumerate}
        \item $5$
        \item $15$
        \item $25$
        \item $35$
    \end{enumerate}
    \item Consider $Z_24$ as the additive group modulo $24$. Then the number of elements of order $8$ in the group $Z_24$ is
    \begin{enumerate}
        \item $1$
        \item $2$
        \item $3$
        \item $4$
    \end{enumerate}
    \item Define $f\colon R^2\to R$ by 
    $$f\brak{x,y}=
    \begin{cases}
        1 \text{, if } xy=0,\\
        2 \text{, otherwise.}
    \end{cases}
    $$
    If $S=\cbrak{\brak{x,y}\colon f\, is\, a\, continuous\, at\, the\, point\, \brak{x,y}}$, then 
    \begin{enumerate}
        \item $S$ is open
        \item $S$ is connected
        \item $S=\phi$
        \item $S$ is closed
    \end{enumerate}
    \item Consider the linear programming problem,\\
    $Max.z=c_1x_1+c_2x_2,c_1,c_2\textgreater0,$\\
    subject to\\
    $x_1+x_3\leq 3\\
    2x_1+3x_2\leq 4\\
    x_1,x_2\geq0$\\
    Then,
    \begin{enumerate}
        \item the primal has an optimal solution but the dual does NOT have an optimal solution
        \item both the primal and dual have optimal solutions
        \item the dual has an optimal solution but the primal does NOT have an optimal solution
        \item neither the primal nor the dual have optimal solutions
    \end{enumerate}
    \item Let $f\brak{x}=x^{10}+x-1,x\in R$and let $x_k=k,k=0,1,2,\dots,10$. Then the value of the divided difference $f\sbrak{x_0,x_1,x_2,x_3,\dots,x_{10}}$ is
    \begin{enumerate}
        \item $-1$
        \item $0$
        \item $1$
        \item $10$
    \end{enumerate}
    \item Let $X$ and $Y$ be jointly distributed random variables having the joint probability density function 
    $$f\brak{x,y}=
    \begin{cases}
        \frac{1}{\pi} \text{,  if } x^2+y^2\leq1,\\
        0 \text{, otherwise},
    \end{cases}
    $$
    Then $P\brak{Y\textgreater max \brak{X,-X}}=$
    \begin{enumerate}
        \item $\frac{1}{2}$
        \item $\frac{1}{3}$
        \item $\frac{1}{4}$
        \item $\frac{1}{6}$
    \end{enumerate}
    \item Let $X_1,X_2,\dots$ be a sequence of independent and identically distributed chi-square random variables, each having $4$ degrees of freedom . Define $S_n=\sum_{i=1}^{n}X_i^2, n=1,2,\dots$.\\
    If $\frac{S_n}{n}\overset{\text{P}}{\longrightarrow}\mu,$ as $n\to \infty$, then $\mu=$
    \begin{enumerate}
        \item $8$
        \item $16$
        \item $24$
        \item $32$
    \end{enumerate}
    \item Let $E_n\colon n=1,2,\dots$ be a decreasing sequence of Lebesgue measurable sets on $R$ and let $F$ be a Lebesgue measurable set on $R$ such that $E_i\cap F=\phi$. Suppose that $F$ has Lebesgue measure $2$ and the Lebesgue measure of $E_n=\frac{2n+2}{3n+1},n=1,2,\dots$.\\
    Then the Lebesgue measure of the set $\brak{\cap_{n=1}^{\infty} E_n}\cup F$ equals
    \begin{enumerate}
        \item $\frac{5}{3}$
        \item $2$
        \item $\frac{7}{3}$
        \item $\frac{8}{3}$
    \end{enumerate}
    \item The extremum for the variational problem
    $$\int_{0}^{\frac{\pi}{8}}\brak{\brak{y\prime}^2+2yy\prime-16y^2}\, dx, y\brak{0}=0,y\brak{\frac{\pi}{8}}=1,$$
    occurs for the curve 
    \begin{enumerate}
        \item $y=\sin\brak{4x}$
        \item $y=\sqrt{2}\sin\brak{2x}$
        \item $y=1-\cos\brak{4x}$
        \item $y=\frac{1-\cos\brak{8x}}{2}$
    \end{enumerate}
\end{enumerate}
\end{document}
